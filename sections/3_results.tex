\section{Experiments}\label{sec:experiments}

%\lau{Put pas tense everywhere.}

\mdeff{If you agree, update throughout (also in figures): $OR \rightarrow L_\text{OR}, E \rightarrow E_\text{OR}$, and use $L_\text{DE}$.}

To evaluate our method, we started with the orientation recovery experiments that included tests of feasibility and sensitivity to distance estimation error.
We then learned the distance using a SiameseNN and compared its performance with the baseline.
%After developing and tuning the architecture of the network for the distance estimation, we continue by introducing the perturbation to the projections.  and the network architecture is adjusted
The robustness of the network to perturbations is then evaluated.
Finally, we ran the whole machine learning pipeline to recover the orientations from estimated distances.

\subsection{Experimental conditions}\label{sec:results:data}

\paragraph{Proteins.}
We considered two proteins (\figref{pdb-proteins}): the $\beta$-galactosidase, a protein with a dihedral (D2) symmetry, and the lambda excision HJ intermediate (HJI), an asymmetric protein with local cyclic (C1) symmetry.
Their deposited PDB atomic models are \texttt{5a1a}~\cite{bartesaghi2015betagal} and \texttt{5j0n}~\cite{laxmikanthan2016structure}, respectively.
For each atomic model, we generated the ground truth by fitting a 5\AA\ density map in Chimera~\cite{pettersen2004ucsf}, which gave us a volume of $110 \times 155 \times 199$ voxels for the $\beta$-galactosidase, and a volume of $69 \times 57 \times 75$ voxels for the HJI.

\begin{figure}[ht!]
    \centering
    \begin{minipage}[b]{0.55\linewidth}
        \centering
        \begin{subfigure}[b]{0.49\linewidth}
            \centering
            \includegraphics[height=5cm]{figures/5a1a_pdb.png}
            \caption*{\texttt{5a1a}}
        \end{subfigure}
        \hfill
        \begin{subfigure}[b]{0.42\linewidth}
            \centering
            \includegraphics[height=5cm]{figures/5j0n_pdb.png}
            \caption*{\texttt{5j0n}}
        \end{subfigure}
        \caption{%
            The ground-truth proteins: $\beta$-galactosidase (\texttt{5a1a})~\cite{5a1a_pdb}, and lambda excision HJ intermediate (HJI) (\texttt{5j0n})~\cite{5j0n_pdb}.
        }\label{fig:pdb-proteins}
    \end{minipage}
    \hfill
    \begin{minipage}[b]{0.35\linewidth}
        \centering
        \begin{subfigure}[b]{0.49\textwidth}
            \centering
            \includegraphics[width=0.8\linewidth]{figures/5j0n_noise0}
            \caption*{$\mathbf{P}_{\bth} \mathbf{x}$}
        \end{subfigure}
        \hfill
        \begin{subfigure}[b]{0.49\linewidth}
            \centering
            \includegraphics[width=0.8\linewidth]{figures/5j0n_noise16}
            \caption*{$\mathbf{P}_{\bth} \mathbf{x} + \mathbf{n}$}
    %, \; \mathbf{n} \sim \mathcal{N}(0, 16\mathbf{I})$}
        \end{subfigure}
        \\ \vspace{1em}
        \begin{subfigure}[b]{0.49\linewidth}
            \centering
            \includegraphics[width=0.8\linewidth]{figures/5j0n_translated}
            \caption*{$\mathbf{S}_{\mathbf{t}} \mathbf{P}_{\bth} \mathbf{x}$}
        \end{subfigure}
        \hfill
        \begin{subfigure}[b]{0.49\linewidth}
            \centering
            \includegraphics[width=0.8\linewidth]{figures/5j0n_noise16_translated}
            \caption*{$\mathbf{S}_{\mathbf{t}} \mathbf{P}_{\bth} \mathbf{x} + \mathbf{n}$}
        \end{subfigure}
        \caption{%
            Example projections of \texttt{5j0n}.
            % (a)~unperturbed, (b)~noisy, (c)~shifted, (d)~noisy and shifted.
        }\label{fig:different-projections}
    \end{minipage}
\end{figure}

\paragraph{Projections.}
From these ground truths, we generated $5,000$ synthetic projections of size $275\times 275$ and $116\times 116$, respectively, using the ASTRA projector~\cite{van2015astra}.
Our projection generator supports two orientation samplings: (i) sampling the Euler angles $\bth=(\theta_3,\theta_2,\theta_1)$ 
%\mdeff{Consistency: we used $\bth=(\theta_3,\theta_2,\theta_1)$ before. Which is better? Motivation for $(\theta_3,\theta_2,\theta_1)$ is that $\theta_1$ is the first rotation that is applied, through $\mathbf{R}_{\theta_1}$.} uniformly, and (ii) sampling uniformly on $\SO(3)$.
Due to protein symmetries, orientations were sampled differently.
The entire \texttt{5j0n} complex being asymmetric~\cite{doi:10.1002/9780470514160.ch4} makes it sufficient to sample the \textit{half} of the $\mathbb{S}^2$ sphere, since the other half will have equivalent projections that are symmetric to the center of this sphere.
Conversely, the $\beta$-galactosidase has D2 symmetry, i.e., it is composed of four identical sub-units with two rotations of magnitude $\pi$ radians around the first axis followed by $\pi$ radians rotation around second axis, as illustrated and explained in~\cite{symmetry_in_protein,symmetry,scipion-em-github, rcsb-symmetry-view, EmpereurMot2019GeometricDO}.
\mdeff{Do we need so many refs? Please check if they are all relevant.}
Therefore, we restricted the sampling to the quarter of the $\mathbb{S}^2$ sphere.
\figref{different-projections} shows example projections.
\mdeff{We should make it clear which 2 Euler angles parameterize $\mathbb{S}^2$, and which remaining one is to parameterize the full $\SO(3)$. Then we could be explicit and write something like we (uniformly?) sampled $(\theta_2, \theta_1) \in [0, \pi[ \times [0, \pi[ \subset [0, \pi[ \times [0, 2\pi[$.}

\todo{As opposed projections are mirrored, we cannot resolve chirality.\footnote{An object is chiral if it cannot be superposed on its mirror image by any combination of translations or rotations.}
Global orientation is lost by projecting, and chirality is lost by integrating.
%\mdeff{Not only a rotation, but an integration through $z_3$. (As opposed projections are mirrored, we cannot resolve chirality. Projecting looses global orientation, integrating looses chirality.)}
That's why we train on half coverage.}
\mdeff{So this should be integrated above.}

\paragraph{Perturbations.}
We considered the following perturbations to control the difficulty of orientation recovery: (i) additive white noise, (ii) translations \mdeff{consistency: shifts or translations?}, (iii) inclusion of the effects of the point-spread functions (PSF).
\mdeff{Did we actually perform experiments with PSF?}
The mathematical formulation of these three components is given in \eqnref{imaging-model}.
\figref{different-projections} shows example perturbations.
\mdeff{Could we show the effect of the PSF?}

\begin{table}[ht!]
    \centering
    \begin{tabular}{lrrr}
        \toprule
        Dataset & Number of projections $P$ (\%) & Maximum number of pairs $P^2$ & Used number of pairs \\
        \midrule
        Train & 2512 (50\%) & 6,312,656 & 63,126 (1\%) \\
        Validation & 1650 (33\%) & 2,722,500 & 27,225 (1\%) \\
        Test & 838 (17\%) & 701,406 & all (sampled per batch) \\
        \bottomrule
    \end{tabular}
    \caption{
        Split of $P=5000$ projections (for both \texttt{5j0n} and \texttt{5a1a}) in training, validation, and test sets.
    }\label{tab:dataset}
\end{table}

\paragraph{Distance learning.}
We used the supervised learning where the input are pairs of images and the output is their respective quaternion distance calculated from the ground truth orientations.
For training, we split our projection dataset into distinct training, validation, and test sets (\tabref{dataset}).
The total number of generated projections was $P = 5000$.
Therefore, the number of possible projection pairs was $P^2 = 25 \times 10^6$.
Splitting $P^2$ into the training, validation, and testing sets would mean that some of the projections appearing in pairs in the training dataset can appear in pairs in the other datasets.
To ensure that the results generalize to unseen data, we split the projections $P$ (and not $P^2$) into training, validation, and testing projection sets.
With these three projections sets we create disjoint projection pair datasets sets (column with $P^2$ values in \tabref{dataset}).
In addition to this, we use only $1\%$ of the possible pairs (last column in the \tabref{dataset}) due to limitation of available resources for the training.
%\todo{Better explain why projections (and not pairs) must be separated in the various sets.}

\paragraph{Orientation recovery.}
Orientations were recovered through \eqnref{orientation-recovery} in a stochastic setting, with the loss function varying over the batches.
To ensure unbiased model, the dataset used in this part was test set from \tabref{dataset}.
Orientation recovery was performed on projections unseen during distance learning.
Since mean orientation error is used in the pose estimation tasks and it is considered reliable performance metric, we decided to use it as our performance measure (see \apxref{metrics-review}). The average difference between predicted and actual angles is degree (or rad), which makes it an intuitive comparison metric.
We employ the definition of a correct estimation: the estimation must be within 10\degree~(0.174 rad) of true orientations for noiseless data and within 25\degree~(0.436 rad) for noisy data.
\mdeff{Do we use this definition?}
\figref{5j0n-aa-loss-perfect-distances} shows a successful convergence and mean orientation recovery error before and after alignment with the perfect distance $d_q$.

%\mdeff{Why is it good? Intuitive sure. Laurène, can we say something more?}

%%%%%%%%%%%%%%%%%%%%%%%%%%%%%%%%%%%%%%%%%%%%%%%%%%%%%%%%%%%%%%%%%%%%%%%%%%%%%%%%%%%%%%%
%\subsection{Results}\label{sec:results:orientation-recovery}



%\mdeff{Story: good distance estimation = good orientation recovery.}
% \begin{algorithm}[H]
% \SetAlgoLined
% \KwResult{Write here the result }
%  initialization\;
%  \For{$steps \gets 1$ \textbf{to} $30000$}{
%   instructions\;
%   \eIf{condition}{
%   instructions1\;
%   instructions2\;
%   }{
%   instructions3\;
%   }
%  }
%  \caption{Orientation recovery algorithm}
% \end{algorithm}


%\subsubsection{Robustness of Recovery to Additive Errors on the Relative Distances}

%%%%%%%%%%%%%%%%%%%%%%%%%%%%%%%%%%%%%%%%%%%%%%%%%%%%%%%%%%%%%%%%%%%%%%%%%%%%%%%%%%%%%%%

\subsection{Sensitivity of orientation recovery to errors in distance estimation}\label{sec:results:orientation-recovery:sensitivity}

%\mdeff{Story: (i) orientation recovery error is strongly linked to distance estimation error, (ii) recovery loss is a good proxy of mean recovery error.}

To prove that orientation recovery is feasible, we first evaluate the performance assuming we have ideal distance metric between two projections, i.e., the quaternion distance between their corresponding orientations.
The method successfully recovers the orientation of every projection, see \apxref{results:orientation-recovery:exact}.

We now go one step further and evaluate the behaviour of~\eqnref{orientation-recovery} when the true relative distances are corrupted by additive Gaussian noise.
The experimental conditions are the same as in the previous section, except that we add an error with increasing variance on the relative distances prior to the minimization.
Precisely: $\widehat{d_p} = d_q + n$, with $n$ sampled from a Gaussian distribution with mean 0 and variances in $[0.0, 0.8]$.
The results are presented in \figref{perfect-with-noise-ar-aa} (red curve).
For all variances, the mean orientation recovery error $E$ is reported in \figref{perfect-with-noise-ar-aa} (blue curve).

\begin{figure}[ht!]
    \centering
    \begin{subfigure}[b]{0.48\textwidth}
        \includegraphics[height=5cm]{figures/5j0n_perfect_noisy_ar_aa}
        \caption{Asymmetric protein (\texttt{5j0n}).}
    \end{subfigure}
    \hfill
    \begin{subfigure}[b]{0.50\textwidth}
    \centering
        \includegraphics[height=5cm]{figures/5a1a_perfect_noisy_ar_aa}
        \caption{Symmetric protein (\texttt{5a1a}).}
    \end{subfigure}
    \caption{
        The mean orientation recovery error $E$ from \eqnref{orientation-recovery-error} is a monotonic function of the distance estimation error.
        Better distance estimation leads to better orientation recovery.
        Moreover, the recovery loss \eqnref{orientation-recovery} is a good proxy for the recovery error $E$, allowing us to assess recovery performance even without ground-truth orientations.
}
    \label{fig:perfect-with-noise-ar-aa}
\end{figure}

These results demonstrate that the performance of orientation recovery~\eqnref{orientation-recovery} depends on the quality of the estimated distances, which advocates for a proper and extensive training of the SiameseNN in the next stages of development.
Another interesting output of \figref{perfect-with-noise-ar-aa} is that it indicates that the error of the orientation recovery behaves as a monotonic function of its loss.
Hence, it suggests that the loss can be used as a good indicator of its performance, which has obvious practical implications for our future works on real data.

%%%%%%%%%%%%%%%%%%%%%%%%%%%%%%%%%%%%%%%%%%%%%%%%%%%%%%%%%%%%%%%%%%%%%%%%%%%%%%%%%%%%%%%

\subsection{Learned function for relative distance estimation }\label{sec:results:distance-estimation:learned}

%\mdeff{Story: learned distance $d_{ps}$ estimates $d_q$ with some variance but still underestimates larger distances.
%Again symmetric vs asymmetric.}

We present here a preliminary evaluation of the ability of SiameseNNs to learn a projection distance $\widehat{d_p}$ that correctly approximates the orientation distance $d_q$.
To assess the progress and effectiveness of our distance estimation implementation, we used an Euclidean distance as a baseline (see \apxref{results:distance-estimation}).

%SiameseNNs come with a variety of more or less powerful architectures.
%At the current stage of development, we work with a simple one.
%Our SiameseNN is composed of two convolutional neural networks (CNNs) with shared weights.
%Their output features vectors are compared through an Eulidean distance, \textit{i.e.}, $d_f(\mathbf{f}_i,\mathbf{f}_j)=\lVert \mathbf{f}_i-\mathbf{f_j}\rVert_2$ in \figref{schematic:distance-learning}.
%Besides the Euclidean distance, this distance metric $F$ can be defined as geodesic distance, or it could be parametrized as MLP, used for a general function approximation, which we will explore in some of the following experiments.
%\mdeff{Don't repeat what's written in \secref{method:distance-learning}. The general stuff goes there, the specific here.}

\begin{figure}
    \centering
    \begin{subfigure}[t]{0.45\textwidth}
        \includegraphics[height=3.5cm]{figures/de_loss_dPdQ_5j0n.pdf}
        \caption{Asymmetric protein (\texttt{5j0n}).}
        \label{fig:losses-siamese-assym}
    \end{subfigure} \quad \quad
    \begin{subfigure}[t]{0.5\textwidth}
        \includegraphics[height=3.5cm]{figures/de_loss_dPdQ_5a1a.pdf}
        \caption{Symmetric protein (\texttt{5a1a}).}
        \label{fig:losses-siamese-sym}
    \end{subfigure}
    \caption{
        Distance learning loss \eqnref{distance-learning} evaluated on the training and validation datasets during learning/training (on the left side respectively). Relationship between orientations' distance $d_q$ and estimated distance $d_p$ on the test dataset (on the right side respectively).
    }\label{fig:losses-siamese}
\end{figure}

% \begin{figure}
%     \centering
%     \begin{subfigure}[b]{0.5\columnwidth}
%         \includegraphics[height=6cm]{figures/dPdQ_5j0n}
%         \caption{Asymmetric protein (\texttt{5j0n}) on test dataset.}
%     \end{subfigure}
%     %\hfill
%     \begin{subfigure}[b]{0.45\columnwidth}
%     \centering
%         \includegraphics[height=6cm]{figures/dPdQ_5a1a}
%         \caption{Symmetric protein (\texttt{5a1a}) on test dataset.}
%     \end{subfigure}
%     \caption{Relationship between orientations' distance $d_q$ and estimated distance $d_p$.}
%     \label{fig:learned-distance-siamese}
% \end{figure}

For each protein, we trained the SiameseNN on its training dataset for 150 epochs ($\sim$2.6 hours) using an RMSProp optimizer~\cite{tieleman2012rmsprop}, a learning rate of $10^{-3}$, and a batch size of 256 pairs.
As a feature distance $d_f$ between the outputs of the two CNNs we used the Geodesic distance \eqnref{geodesic-distance}.
\mdeff{We wrote in the method that we take $d_f=d_q$. Shall we repeat? I don't think we should if the only experiments where that is untrue is in \apxref{de-influence-arch}.}
The pairs for the training were sampled from $1\%$ of maximum number of training pairs $P_{\text{train}}^2$ ($63,126$ pairs) and validation was performed on $1\%$ of the maximum number of validation pairs $P_{\text{val}}^2$ ($27,225$ pairs).
We limited our training and validation dataset due to Google Colaboratory\footnote{A hosted Jupyter notebook service from Google Research with resources that are not guaranteed and not unlimited.} training time limit of 12 hours.

%\mdeff{So those pairs are sampled from $63,126$ pairs from the training dataset, rather than the $P^2$ possible pairs? If true, we should motivate somewhere why we limit our training dataset.}
Depending on the available resources on the Google Colaboratory, the training lasted from 2.6 hours to 9.3 hours on one of its GPUs.
The evolution of the training and validation losses are presented on the left side in \figref{losses-siamese-assym} for the asymmetric protein (\texttt{5j0n}), and on the left side in \figref{losses-siamese-sym} for the symmetric one (\texttt{5a1a}).
The results demonstrate that the SiameseNN succeeded at learning a proxy distance for the asymmetric protein dataset, as convergence was reached in about 50 epochs ($\sim$ 50 minutes in the best resource availability setting).

Similarly to Euclidean distance $d_p$, we noticed that the larger distances $d_q$ were poorly predicted and the plot again had a slight plateau phenomenon for the distances higher than~$2.5$ rad.
\mdeff{Insist that this is much better than Euclidean $d_p$.}
\mdeff{Future work: fix or downplay the influence of larger distances in orientation recovery.}

It is interesting to see that both validation losses were around $0.2$.
However, the current SiameseNN architecture slightly overfits at learning the distance for the dataset \texttt{5a1a}, which is very likely due to the symmetry of the $\beta$-galactosidase protein, even thought the quarter-sphere coverage was used.
%Indeed, its synthetic dataset may still contain pairs of projections that share the same $d_p$, yet differ in their $d_q$.
This simply advocates for the restriction to non-overlapping areas on $\SO(3)$ when sampling the orientations used to generate the SiameseNN training dataset.
The latter would then only contain projection pairs with a linear $(d_q,d_p)$ relationship, which should ensure a successful training of the network.
%\mdeff{I don't get this explanation. Do you mean that \texttt{5a1a} might have other symmetries than D2?}
For the rest of the experiments, we used the asymmetric protein (\texttt{5j0n}) dataset.
Besides using the asymmetric protein, we performed the full protein reconstruction pipeline on the symmetric protein (\texttt{5a1a}).

We then fed to the trained SiameseNN $1,000$ pairs of projections randomly selected from the \texttt{5j0n} testing dataset, and reported the $(d_q,\widehat{d_p})$ relationship of each pair in \figref{losses-siamese} (right side of each subfigure).
These results confirm that, the SiameseNN was able to predict the orientation distance $d_q$ using only the projections as inputs.
The prediction performance was slightly better for the asymmetric protein compared to symmetric protein.
Moreover, it clearly outperformed the Euclidean distance at doing so.
These preliminary results are encouraging, as much has yet to be gained from improving upon the rather primitive SiameseNN architecture we currently use.
The architecture of implemented SiameseNN can be seen in \apxref{siamese-architecture}.
Besides concentrating on the hyperparameter selection and tuning of the neural network layers, we also evaluated the performance of the model depending on the feature distance we used for the SiameseNN, see \figref{geo-eucl-mlp} and \apxref{de-influence-arch}.

%%%%%%%%%%%%%%%%%%%%%%%%%%%%%%%%%%%%%%%%%%%%%%%%%%%%%%%%%%%%%%%%%%%%%%%%%%%%%%%%%%%%%%%

\subsection{Sensitivity of distance learning to perturbations in the projections}\label{sec:results:distance-estimation:sensitivity}

%\mdeff{Story: learned distance is minimally sensible to perturbations (additive noise, translation, PSF) because we can train it to ignore irrelevant information.
%Thanks again to good model of cryo-EM imaging.}
%\mdeff{Better word? (perturbations, corruptions, quality, non-ideal)}

% Intro and shift.
We desire to estimate distances that are invariant to perturbations in the projections, specified by~\eqnref{imaging-model}.
As discussed in \secref{method:distance-learning}, the convolutional architecture of the SiameseNN should be shift invariant.
\figref{results:distance-estimation:shift} indeed shows that learning distances (and hence recovering orientations) is insensible to shifts in projections.

% Noise.
As we cannot---or do not (yet) know how to---build noise invariance into the architecture, we trained the SiameseNN on noisy datasets and evaluated whether it could learn to ignore noise as being irrelevant information.
% brute force vs principled engineering
\figref{results:distance-estimation:noise} shows a mean orientation recovery error of $E_\text{OR} \approx 0.16$ radians ($5\degree$) for noiseless projections and $E_\text{OR} \approx 0.42$ radians ($13\degree$) for a more realistic noise variance of $\sigma^2=15$.
\mdeff{How good is that?}
While not invariant, the SiameseNN learned to discard noise (while an Euclidean distance would be tremendously sensible to it).
Moreover, overfitting (i.e., the growing gap between the validation and trainig losses) indicates that more training data will decrease noise sensitivity.
\mdeff{Here or future work?}

% PSF and conclusion.
We didn't evaluate sensitivity to the PSF but expect a similar behavior.
%While we would ideally want the NN architecture to be engineered to ignore irrelevant information, that is not always possible.
%We know how to do it for shifts, not noise or PSF.

Note that we observe again (\secref{results:orientation-recovery:sensitivity}) that an higher recovery loss $L_\text{OR}$ induces an higher error $E_\text{OR}$, and that the estimation of more accurate distances (a smaller $L_\text{DE}$) induces the recovery of more accurate orientations (a smaller $L_\text{OR}$ and $E_\text{OR}$).

\begin{figure}[ht!]
    \centering
    \begin{subfigure}[t]{0.47\linewidth}
        \includegraphics[width=\linewidth]{figures/de_translation_nums}
        \caption{%
            Learning from shifted projections $\{ \mathbf{S}_{\mathbf{t}_i} \mathbf{P}_{\bth_i} \mathbf{x} \}$, with translations $t_{i_1}$ and $t_{i_2}$ sampled from a triangular distribution with mean 0 and of increasing limits.
            Learning is not harder as projections get shifted farther, because shift invariance is built into the convolutional architecture of $\mathcal{G}_w$.
    }\label{fig:results:distance-estimation:shift}
    \end{subfigure}
    \hfill
    \begin{subfigure}[t]{0.47\linewidth}
        \includegraphics[width=\linewidth]{figures/de_noises_nums}
        \caption{%
            Learning from noisy projections $\{ \mathbf{P}_{\bth_i} \mathbf{x} + \mathbf{n} \}$, with white noise $\mathbf{n} \sim \mathcal{N}(0, \sigma^2\mathbf{I})$ of increasing variance $\sigma^2$.
            Learning is harder as projections get noisier, because noise invariance is not built into the architecture of $\mathcal{G}_w$.
        }\label{fig:results:distance-estimation:noise}
    \end{subfigure}
    \caption{%
        Sensitivity of distance learning to perturbations in the projections of \texttt{5j0n}.
        The box plots show the distance learning loss $L_\text{DE}$ \eqnref{distance-learning} on the training (blue) and validation (red) sets.
        Boxes show the orientation recovery loss $L_\text{OR}$ \eqnref{orientation-recovery} and error $E_\text{OR}$ \eqnref{orientation-recovery-error}.
        \mdeff{It's confusing that red and blue are used for both train vs validation and $E$ vs $OR$. I don't think we need colors for $E$ and $OR$. They could both be in the same black box.}
        \mdeff{Legend consistency: train -> training set, validation -> validation set.}
    }
\end{figure}

%%%%%%%%%%%%%%%%%%%%%%%%%%%%%%%%%%%%%%%%%%%%%%%%%%%%%%%%%%%%%%%%%%%%%%%%%%%%%%%%%%%%%%%

\subsection{Orientation recovery from estimated distances}

\mdeff{We recovered orientations in the last section too.}

%\mdeff{Story: pipeline works but better distance estimation is needed for SOTA reconstruction.
%Method is however promising because learned distance is robust to perturbations and recovery works if distance works.}
%\todo{Justify threshold because of plateau (figref).
%Show recovered orientations w.r.t.\ ground truth after alignment.}
%\todo{Reconstruct the protein to show the full pipeline: from a set of projections to a reconstructed protein.
%Emphasize that it's a naive reconstruction algorithm.}

The orientation recovery from estimated distances represents a full pipeline needed to reconstruct the protein from a given set of projections.
We ran the pipeline for both, asymmetric (\texttt{5j0n}) and symmetric (\texttt{5a1a}) protein.
In addition, we ran the pipeline for the simulated realistic noise in the asymmetric protein.
The experimental setting for distance estimation was similar to the one used to generate the \figref{losses-siamese}: 150 epochs, 1e-3 learning rate, batch size 256 with random sampling of the projections, but for feature distance metric we used the geodesic distance since it showed the best performance in \figref{geo-eucl-mlp}.

We then ran the orientation recovery on the estimated distances of the asymmetric protein and the performance results can be observed in \figref{5j0n-orientation-recovery-loss-est} with the same experimental setting as in \figref{5j0n-orientation-recovery-loss}.
With the noiseless projections, the objective function successfully converged to the $0.0510$ and with the noisy projections, the objective function converged to the $0.0683$.

\begin{figure}[ht!]
    \centering
    \begin{subfigure}[b]{0.45\linewidth}
        \includegraphics[height=5.5cm]{figures/5j0n_noise0_ar_aa}
        \caption{Recovery from noiseless projections $\{ \mathbf{P}_{\bth_i} \x \}$.}
    \end{subfigure}
    \hfill
    \begin{subfigure}[b]{0.51\linewidth}
    \centering
        \includegraphics[height=5.5cm]{figures/5j0n_noise16_ar_aa}
        \caption{Recovery from noisy projections $\{ \mathbf{P}_{\bth_i} \x + \mathbf{n} \}, \; \mathbf{n} \sim \mathcal{N}(0, 16\mathbf{I})$.}
    \end{subfigure}
    % \\
    % \begin{subfigure}[b]{0.45\textwidth}
    % \centering
    %     %\includegraphics[height=5.7cm]{figures/5j0n_noise0_angle_alignment_before}
    %     \includegraphics[height=5.7cm]{figures/5j0n_noise0_angle_alignment_after}
    %     \caption{Recovery error, noiseless projections $\mathbf{Px}$.}
    %     \label{fig:angle-alignment-5j0n-noise0}
    % \end{subfigure}
    % \hfill
    % \begin{subfigure}[b]{0.5\textwidth}
    % \centering
    %     %\includegraphics[height=5.7cm]{figures/5j0n_noise16_angle_alignment_before}
    %     \includegraphics[height=5.7cm]{figures/5j0n_noise16_angle_alignment_after}
    %     \caption{Recovery error, noisy projections $\mathbf{Px+n}, \; \mathbf{n} \sim \mathcal{N}(0, 16\mathbf{I})$.}
    %     \label{fig:angle-alignment-5j0n-noise16}
    % \end{subfigure}
    \caption{%
        Performance of orientation recovery of the asymmetric protein (\texttt{5j0n}).
        The blue curve shows the recovery loss~\eqnref{orientation-recovery}, with the minimum highlighted.
        The red histogram shows the recovery error~\eqnref{orientation-recovery-error}, with the mean $E$ highlighted.
    }\label{fig:5j0n-orientation-recovery-loss-est}
\end{figure}

The mean orientation recovery error for asymmetric protein \texttt{5j0n} without noise in the projection is shown in \figref{5j0n-orientation-recovery-loss-est}~\textbf{(a)}.
The smallest error achieved is $0.1594$ rad.
The mean orientation recovery error for asymmetric protein \texttt{5j0n} with noisy projections (white noise with variance 16) is shown on histogram in \figref{5j0n-orientation-recovery-loss-est}~\textbf{(b)}.
The smallest error achieved is $0.4184$ rad.

As a last step of the pipeline, we performed protein reconstruction using the projections and their corresponding estimated orientations.
Using the ASTRA toolbox, we generated orientation vectors based on angles which we fed into projection 3D geometry in ASTRA.
With total of $1,650$ projections in the test dataset, we were able to reconstruct the protein.
The reconstruction results for the asymmetric protein (\texttt{5j0n}) with \textit{noiseless projections} are shown in \figref{5j0n-reconstruction}: \textbf{(a)} using the ground-truth orientations and \textbf{(c)} using the estimated aligned orientations.
The reconstruction results for the asymmetric protein (\texttt{5j0n}) with \textit{noisy projections} are shown in \figref{5j0n-reconstruction}: \textbf{(b)} using the ground-truth orientations and \textbf{(d)} using the estimated aligned orientations.

\begin{figure}[ht!]
    \centering
    \begin{subfigure}[b]{0.22\linewidth}
        \centering
        \includegraphics[width=0.99\linewidth]{figures/5j0n_reconstruction_GT}
        \caption{}
    \end{subfigure}
    \hfill
    \begin{subfigure}[b]{0.22\linewidth}
        \centering
        \includegraphics[width=0.99\linewidth]{figures/5j0n_reconstruction_GT_noise16}
        \caption{}
    \end{subfigure}
    \hfill
    \begin{subfigure}[b]{0.22\linewidth}
        \centering
        \includegraphics[width=0.99\linewidth]{figures/5j0n_reconstruction_noise0}
        \caption{}
    \end{subfigure}
    \hfill
    \begin{subfigure}[b]{0.22\linewidth}
        \centering
        \includegraphics[width=0.99\linewidth]{figures/5j0n_reconstruction_noise16}
        \caption{}
    \end{subfigure}
    \caption{
        Performance of orientation recovery of the asymmetric protein (\texttt{5j0n}). \textbf{(a)} Noiseless projections $\mathbf{Px}$, true orientations ${\big\{q_p\big\}}_{p=1}^P$. \textbf{(b)} Noisy projections $\mathbf{Px + n}$, true orientations ${\big\{q_p\big\}}_{p=1}^P$. \textbf{(c)} Noiseless projections $\mathbf{Px}$, recovered orientations ${\big\{\widehat{q_p}\big\}}_{p=1}^P$. \textbf{(d)} Noisy projections $\mathbf{Px + n}$, recovered orientations ${\big\{\widehat{q_p}\big\}}_{p=1}^P$.
    }\label{fig:5j0n-reconstruction}
\end{figure}

Similarly, we ran the whole reconstruction pipeline on the symmetric protein (\texttt{5a1a}).
The experimental conditions for the distance estimation were the same as for the asymmetric protein, except that we used the quarter-sphere projections coverage (whereas, in the asymmetric protein we used half-sphere coverage).
The orientation recovery loss is shown in \figref{5a1a-orientation-recovery-loss}.
It successfully converged to $0.0381$.
The mean orientation recovery error for symmetric protein \texttt{5a1a} is shown in \figref{angle-alignment-5a1a-noise0}.
The smallest error achieved was $0.1871$ rad.

\begin{figure}[ht!]
    \centering
    \begin{subfigure}[b]{0.45\textwidth}
        \centering
        \includegraphics[height=5.5cm]{figures/5a1a_noise0_ar_aa}
        \caption{Recovery loss \eqnref{orientation-recovery} and recovery error \eqnref{orientation-recovery-error}.}
    \end{subfigure}
    \hfill
    \begin{subfigure}[b]{0.25\textwidth}
        \centering
        \includegraphics[width=0.99\linewidth]{figures/5a1a_ground_truth}
        \caption{Reconstruction from true orientations ${\big\{q_p\big\}}_{p=1}^P$.}
    \end{subfigure}
    % \\
    % \begin{subfigure}[b]{0.45\textwidth}
    %     \centering
    %     %\includegraphics[height=5.7cm]{figures/5a1a_noise0_angle_alignment_before}
    %     \includegraphics[height=5.5cm]{figures/5a1a_noise0_angle_alignment_after}
    %     \caption{Orientation recovery error \eqnref{orientation-recovery-error}.}
    % \end{subfigure}
    \hfill
    \begin{subfigure}[b]{0.25\textwidth}
        \centering
        \includegraphics[width=0.99\linewidth]{figures/5a1a_aligned}
        \caption{Reconstruction from recovered orientations ${\big\{\widehat{q_p}\big\}}_{p=1}^P$.}
    \end{subfigure}
    \caption{
        Orientation recovery and reconstruction of the symmetric protein (\texttt{5a1a}) from noiseless projections.
    }\label{fig:5a1a-orientation-recovery-loss}
    \label{fig:angle-alignment-5a1a-noise0}
    \label{fig:5a1a-reconstruction-noise0}
\end{figure}

Lastly, we performed the protein reconstruction with the same ASTRA toolbox setting as for the asymmetric protein.
The results of the reconstruction are shown in \figref{5a1a-reconstruction-noise0}.
We successfully reconstructed the symmetric protein even though the distance estimation was noisier than the one performed on the asymmetric protein.

We observe that the pipeline works, but for the state-of-the-art reconstruction we need a better distance estimation.
However, the method developed is promising since the learned distance is robust to perturbations.
We observe that the orientation recovery and distance estimation are interconnected, i.e., if one works the other one will work.

To check the generalization to unseen proteins, and not only to  unseen projections, we trained the distance estimation on the set of four different proteins (\texttt{5nvu}~\cite{5nvu_pdb}, \texttt{5nvs}~\cite{5nvs_pdb}, \texttt{6mem}~\cite{6mem_pdb}, \texttt{6o1o}~\cite{6o1o_pdb}) that have the same type of symmetry (asymmetric  C1) as the one protein (\texttt{5j0n}) in the test dataset that is used in orientation recovery.
The orientation recovery reached the error of 0.0352.
This proves that distance learning was able to abstract the protein and that way generalized the distance metric to the unseen proteins.
