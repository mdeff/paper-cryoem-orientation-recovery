\clearpage

\section{Different performance metrics}\label{apx:metrics-review}
There are many different ways of evaluating the pipeline performance found in the field of pose estimation. Some of the evaluations include the following:
\mdeff{Shall we keep this? Not as a bullet list for sure.} \lau{Maybe in the Appendix, but not in the main text imo.}
\begin{itemize}
\item Intersection over Union (IoU) of the object 3D cloud with a custom threshold classifying it as a good estimate or not (e.g. in the paper~\cite{10.1007/s11263-014-0733-5} the threshold score above 0.5 is considered good estimation).
\item Translation and rotation error between estimated 3D model and true 3D model with fixed thresholds (e.g. in the paper~\cite{shotton2013scene} they require the translation error to be below 5 cm and rotation error to be below 5\degree)
\item The average distance of all the points of the model from their transformed version, and if the error is less than the constant multiple of diameter of the 3D model, it is considered correctly evaluated (e.g. evaluation error is used in papers \cite{10.1007/978-3-642-37331-2_42, xiang2018posecnn})
\item Reprojection error that projects the estimated points onto the image and computes the pairwise distances in the image space, instead of computing distances in the 3D model space (e.g. used in paper~\cite{xiang2018posecnn})
\item The recovery error measured as Frobenius norm from estimated 3D model and true model, where 3D model is composed of 3D locations of important landmarks (e.g. elbow for human pose estimation)~\cite{wangni2018monocular}
\item Average Orientation Similarity (AOS) is the difference between the true and estimated model with a cosine similarity term~\cite{RedondoCabrera2016PoseEE}
\item Mean Angle Error (MAE) and Median Angle Error (MedError) evaluated and compared with other pose estimation error metrics in the paper~\cite{RedondoCabrera2016PoseEE}.
\end{itemize}

\section{Orientation recovery: Recovery from exact distances}\label{apx:results:orientation-recovery:exact}

%\mdeff{Story: works perfectly despite no convexity guarantee and sampling.}
%\mdeff{I made it concise but precise. Let's do that for all!}

To verify that (i) the lack of a convexity guarantee for \eqnref{orientation-recovery} and (ii) the severe sampling of the sum are non-issues in practice, we attempt orientation recovery under exact distance estimation $d_p(\p_i, \p_j) = d_q(q_i, q_j)$.
Orientations are perfectly recovered.
\figref{5j0n-orientation-recovery-loss} shows the convergence of~\eqnref{orientation-recovery} to zero.

\begin{figure}[ht!]
    \centering
    %\begin{subfigure}[b]{0.45\textwidth}
        \includegraphics[height=7cm]{figures/5j0n_perfect_angle_recovery}
        %\caption{Asymmetric protein (\texttt{5j0n}).}
    %\end{subfigure}
    %\hfill
    %\begin{subfigure}[b]{0.5\textwidth}
    %\centering
    %    \includegraphics[height=5.5cm]{figures/5a1a_perfect_angle_recovery}
    %    \caption{Symmetric protein (\texttt{5a1a}).}
    %\end{subfigure}
    \caption{
        Example of perfect orientation recovery (for \texttt{5j0n}).
        The loss~\eqnref{orientation-recovery} converges to zero when the distance estimation is perfect, i.e., $d_p(\p_i, \p_j) = d_q(q_i, q_j)$.
    }\label{fig:5j0n-orientation-recovery-loss}
\end{figure}

Empirically we find that the mean orientation recovery error \eqnref{orientation-recovery-error} $E = 0$.
The sphere coverage before and after the orientation alignment can be seen in the \figref{5j0n-aa-loss-perfect-distances}.
We can see that the orientation alignment was performed successfully.

\begin{figure}[ht!]
    \centering
    \begin{subfigure}[b]{0.45\textwidth}
        \includegraphics[height=6cm]{figures/coverage_alignment_before}
        \caption{Orientations before alignment.}
    \end{subfigure}
    \hfill
    \begin{subfigure}[b]{0.50\textwidth}
    \centering
        \includegraphics[height=6cm]{figures/coverage_alignment_after}
        \caption{Orientations after alignment.}
    \end{subfigure}
    \\
    \begin{subfigure}[b]{0.45\textwidth}
        \includegraphics[height=5.5cm]{figures/5j0n_perfect_angle_ralignment_before}
        \caption{Orientation recovery error without alignment.}
    \end{subfigure}
    \hfill
    \begin{subfigure}[b]{0.5\textwidth}
    \centering
        \includegraphics[height=5.5cm]{figures/5j0n_perfect_angle_ralignment_after}
        \caption{Orientation recovery error with alignment.}
    \end{subfigure}
    \caption{%
        Example of perfect alignment after a perfect orientation recovery under the true distance $d_p(\p_i, \p_j) = d_q(q_i, q_j)$ (\texttt{5j0n}).
        The first row shows the orientation coverage of $\mathbb{S}^2 \subset \SO(3)$ after the recovery.
        Green points are the ground truth orientations ${\{q_p\}}_{p=1}^P$ and red points are the recovered orientations ${\{\widehat{q_p}\}}_{p=1}^P$. The Figure (b) is exactly aligned which can be seen when zoomed. Due to plotting artifacts, both colors can still be seen.
        % orientations projected on S²
        The second row are the histograms of ${\{ d_q (q_p, \T \widehat{q_p}) \}}_{p=1}^P$, with $\T = \mathbf{I}$ on the left (i.e., without alignment) and $\T$ as the optimum of \eqnref{orientation-recovery-error} on the right.
        Alignment is necessary to evaluate the performance of orientation recovery.
    }\label{fig:5j0n-aa-loss-perfect-distances}
%    \label{fig:angle-alignment-perfect}
\end{figure}

\section{Distance estimation: Euclidean distance}\label{apx:results:distance-estimation}
%\subsection{Estimating Relative Orientations from Projections}
%\subsection{Relative orientation estimation}

%\mdeff{Story: $d_p$ good estimator of $d_q$.
%SiameseNN better than l2, but still plateaus.
%Robust to projection noise.}

The challenge of distance estimation is to define the distance between two projections $d_p$ such that it is connected to the distance between two quaternions $d_q$. We start by using the Euclidean distance as the baseline. Then, we learn the distance metric using the SiameseNN architecture. With this network we aim to classify the new unseen pairs of projections without training the network again. Afterwards, we explore different network architectures as well as test the sensitivity of the network to perturbed projections.

%\subsubsection{Euclidean distance}\label{sec:results:distance-estimation:euclidean}

%\mdeff{Story: simplest baseline estimator, $d_{pe}$ somewhat estimates $d_q$, quickly plateaus (even in the simplest noiseless and centered case).
%Note the difference between symmetric and asymmetric proteins.}

We evaluate $d_p(\p_i, \p_j) = \Vert \p_i - \p_j \Vert_2$ (i.e., the Euclidean distance) as a baseline distance estimator.
From $P = 5,000$ possible projection, we randomly select $5$ projections.
For each of these projections, we compute the Euclidean distance between aforementioned projection and all the others $d_p(\mathbf{p}_i,\mathbf{p}_j)=\lVert\mathbf{p}_i-\mathbf{p}_j\rVert_2$ and their corresponding orientation distance $d_q(q_i,q_j)$ through~\eqnref{distance:orientations}.
We then report the $(d_q,d_p)$ relationship for all pairs in \figref{euclidean-not-robust}, for both the asymmetric protein (left) and the symmetric one (right).

\begin{figure}[ht!]
    \centering
    \begin{subfigure}[t]{0.45\textwidth}
        \includegraphics[height=7.5cm]{figures/eucl_notrobust_5j0n}
        \caption{Asymmetric protein (\texttt{5j0n}).}
    \end{subfigure}
    \hfill
    \begin{subfigure}[t]{0.45\textwidth}
        \includegraphics[height=7.5cm]{figures/eucl_notrobust_5a1a}
        \caption{Symmetric protein (\texttt{5a1a}).}
    \end{subfigure}
    \caption{
        The Euclidean distance between two projections $d_p(\p_i, \p_j) = \Vert \p_i - \p_j \Vert_2$ versus their actual relative orientation $d_q(q_i, q_j)$ for \textbf{(a)} the asymmetric protein (\texttt{5j0n}) dataset, and \textbf{(b)} the symmetric protein (\texttt{5a1a}) dataset.
        The color corresponds to projection pairs that share one projection, i.e., distance between one projection with all other projections.
    }\label{fig:euclidean-not-robust}
\end{figure}

Two principal observations can be made from this experiment.
First, as suspected, $d_p$ fails to be a consistent predictor of $d_q$, even in the simple imaging conditions considered here (no noise, no shift, no PSF).
In particular, the larger the quaternion distance $d_q$, the poorer the predictive ability of $d_p$ (the plot plateaus).
The other interesting observation is that the trend of $(d_q,d_p)$ plot of the $\beta$-galactosidase protein (\texttt{5a1a}) appears to take symmetric shape of letter \texttt{M} which can be explained with the fact that this protein has intrinsic dihedral (D2) symmetry~\cite{noauthor_d2sym_nodate,noauthor_5a1asym_nodate}.
\mdeff{Check these refs. Other ones are used in main text?}

\section{Choice of feature distance}\label{apx:feature-distance}

%\mdeff{Story: $d_f = d_q$ better than Euclidean and MLP $d_f$. }

There are multiple options for a distance function between features $\mathbf{f}_i = \mathcal{G}_w(\p_i) \in \R^{n_f}$.
As we wrote in \secref{method:distance-learning}, we chose the cosine distance
\begin{equation*}
    d_f(\mathbf{f}_i,\mathbf{f}_j) = 2 \arccos \left( \frac{\langle \mathbf{f}_i, \mathbf{f}_j \rangle}{\lVert \mathbf{f}_i \rVert \lVert \mathbf{f}_j \rVert} \right).
\end{equation*}
\mdeff{No absolute value here? As in \eqnref{distance:orientations}.}
A common choice is the Euclidean distance $d_f(\mathbf{f}_i, \mathbf{f}_j) = \| \mathbf{f}_i - \mathbf{f}_j \|_2$.

Another option is to parametrize $d_f$ as a multi-layer perceptron (MLP) and learn it.
We tried with an MLP made of six layers with $1024, 512, 512, 256, 256, 1$ units, all of which use the SeLU activation function.
The features $\mathbf{f}_i$ and $\mathbf{f}_j$ were stacked as an array of size $2n_f$ to be fed to the MLP.

\figref{geo-eucl-mlp} compares the three.
While an MLP can approximate any function, there is no guarantee that the learned function will satisfy the axioms of a proper distance function (i.e., the identity of indiscernibles, symmetry, and triangle inequality).
Given its flexibility, it needs a lot of data to properly train---hence why it failed.

The performances of the Euclidean and cosine distances are close.
The cosine is however best: $L_\text{DE}$ is the lowest, making $d_p$ a better estimator of $d_q$.
% the projection pairs deviate the least from the identity
The cosine distance performs better as it models the elliptic geometry of $\SO(3)$, while the Euclidean distance does not (as Euclidean space is neither periodic nor curved).
%The $\SO(3)$ is non-linear and it can be explained with the fact that Euclidean distance of two quaternions can be small, despite the rotation being large~\cite{huynh_metrics_2009,DBLP:journals/corr/abs-1805-01026}.

\begin{figure}
    \centering
    \includegraphics[height=7cm]{figures/geo_eucl_mlp_distance_metric.pdf}
    \caption{
        Performance of distance learning w.r.t.\ the choice of feature distance function $d_f$.
        The box plots show the distance learning loss $L_\text{DE}$ \eqnref{distance-learning} on the training (blue) and validation (red) sets.
        The inserted plots show the relationship between $d_q(q_i, q_j)$ and $d_p(\p_i, \p_j) = d_f(\mathcal{G}_w(\p_i), \mathcal{G}_w(\p_j))$.
        \todo{Consistency: train -> training set, validation -> validation set. Add $L_\text{DE}$ to the y-axis label. Larger text.}
        \mdeff{Can we make this figure wider but flatter? To better fit the page and use less vertical space.}
    }\label{fig:geo-eucl-mlp}
\end{figure}

% it could be tempting to
An alternative would have been to train $\mathcal{G}_w$ to directly map projections to orientations, i.e., $\widehat{q_i} = \mathbf{f}_i = \mathcal{G}_w(\p_i)$, and avoid the orientation recovery step.
While that solution is simpler, learning an embedding in a space of $n_f \gg 4$ dimensions gives enough room for $\mathcal{G}_w$ to represent different proteins, noise levels, PSFs, etc., which are then abstracted by taking distances.
% $n_f=4$ dimensional space is too constrained
\mdeff{Did we try that? Would be nice to have the results if we did.}

\clearpage
\section{SiameseNN architecture}\label{apx:siamese-architecture}

\begin{figure}[h!]
    \centering
    \includegraphics[height=19cm]{figures/model_plot.png}
    \caption{%
        Distance estimation network architecture.
        We have two input images of dimensions $116 \times 116$.
        Each one goes to its CNN (part where we share the weights).
        The output is a scalar value representing the distance between these two images.
    }\label{fig:de-architecture}
\end{figure}
