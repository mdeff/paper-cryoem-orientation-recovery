\section{Context}
\label{sec:context}

In single-particle analysis (SPA), the 3D particles adopt a random orientation in the ice layer before being imaged with parallel electron beams. Hence, the projection geometry associated to each 2D projection is unknown. Yet, this knowledge is essential for tomographic reconstruction. To handle this, a popular approach used by most SPA software packages is to alternatively refine the 3D structure and the orientation estimation~\cite{penczek1994ribosome,Baker1996,Dempster1977,sigworth1998maximum,scheres2012bayesian}. Iterative refinement procedures are extremely powerful and have permitted the determination of numerous biological structures up to near-atomic resolution~\cite{kuhlbrandt2014resolution}. Unfortunately, the outcome of these methods is predicated on the quality of the initial reconstruction, or, equivalently, on the initial estimation of the orientations~\cite{sorzano2006optimization,henderson2012outcome}. Several methods have been designed to produce a first rough structure~\cite{singer2010detecting,wang2013orientation,greenberg2017common,punjani2017cryosparc,pragier2019common}, but this remains a notoriously arduous challenge in SPA.  

In this work, we present the outline and the preliminary results of an ongoing research project for SPA that capitalizes on the powerful learning capabilities of neural networks, yet still fundamentally relies on our ability to faithfully model the cryo-EM imaging process (for the generation of the training dataset). As we shall shortly detail, its target is the design of a method that learns to estimate the unknown orientation associated to each projection in a SPA dataset without relying on any intermediate reconstruction procedure. The method is still at its proof-of-concept stage, and several interesting developmental steps lie ahead.
