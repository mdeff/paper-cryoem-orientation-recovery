\section{Introduction}

\mdeff{Content. (p1) Why is the general problem of protein reconstruction important and difficult. (p2) Short background on SPA/Cryo-EM and how reconstruction is done. (p3) Previous work on orientation estimation (or initial structure estimation). If there's none (because people have researched other routes), state it and write why it's an interesting route to explore. How does it compare to initial rough structure estimation? Or whatever the other routes are. (p4) Our contribution on top of previous work.}

\todo{Figure about the geometry of the 3D imaging model.}
\mdeff{Maybe better with \secref{method:orientation-representation} to show how orientations are rotations.}

In single-particle analysis (SPA), the 3D particles adopt a random orientation in the ice layer before being imaged with parallel electron beams.
Hence, the projection geometry associated to each 2D projection is unknown.
Yet, this knowledge is essential for tomographic reconstruction.
To handle this, a popular approach used by most SPA software packages is to alternatively refine the 3D structure and the orientation estimation~\cite{penczek1994ribosome,Baker1996,Dempster1977,sigworth1998maximum,scheres2012bayesian}.
Iterative refinement procedures are extremely powerful and have permitted the determination of numerous biological structures up to near-atomic resolution~\cite{kuhlbrandt2014resolution}.
Unfortunately, the outcome of these methods is predicated on the quality of the initial reconstruction, or, equivalently, on the initial estimation of the orientations~\cite{sorzano2006optimization,henderson2012outcome}.
Several methods have been designed to produce a first rough structure~\cite{singer2010detecting,wang2013orientation,greenberg2017common,punjani2017cryosparc,pragier2019common}, but this remains a notoriously arduous challenge in SPA.

In this work, we present a method that learns to estimate the unknown orientation associated to each projection in a SPA dataset without relying on any intermediate reconstruction procedure.
