\section{Method}

% \lau{Lau: As far as I'm concerned, everything below before subsection 2.1 could also go at the end of the Introduction. Beyond the fact that it contains literature review and conceptual information that would go well in the intro, I find the current architecture of Section 2 a bit puzzling. First there is text to explain the methods and detail their intuition/key components, then there are three consecutive subsections that kind of do the same, but from a maths points of view. The aforementioned suggestion of putting the first text in the intro could solve this. Another possibility: put the descriptive texts in their corresponding sub-sections, so that those contain both the motivation and the maths. I'm open to further suggestions and suggestions ;-) }
% \mdeff{Indeed. The "Distance Learning" and "Orientation Recovery" paragraphs might go in their respective sub-sections. But I'd like to keep a high-level motivation of the whole method (the text linked to the second figure to appear alongside \figref{imaging-geometry}.}
% \lau{Fine for me.}
% \banjac{I moved the DE and OR to corresponding subsections. The short intro to methods I left here at the beginning of the section 2 (I didn't move it to intro). Let me know if you meant something different}

\begin{figure}
    \centering
    \includegraphics[width=\linewidth]{schematic_method_overview}
    \caption{%
        \textbf{Method:}
        %Our method consists of two steps:
        (i) we estimate distances between pairs of projections, and (ii) we recover the orientation of each projection from these distances.
        % \mdeff{It would be amazing to have a visual of the embedding made in step 2: finding points in SO(3) such that the geodesic distances between them are the estimated distances. As we can't see SO(3) / $S^3$, we could try with $S^2$, which is a subset/projection.}
        \mdeff{Step 2: $\{q_k \in \mathbb{S}^3 \}$ instead of $q_i, q_j \in \mathbb{H}$ as in \eqnref{orientation-recovery}.}
        \mdeff{The first "Intermediate representation" shouldn't have the $\widehat{q}$ yet. I would keep the "?" as in "Inputs" or have $\p_i$ alone.}
    }\label{fig:schematic:method-overview}
\end{figure}

% Motivation (second figure).
Our approach relies on two observations (\figref{intuition-method}).
First, the greater the similarity between two 2D projections, the more likely they originated from two 3D particles that adopted close orientations in the ice layer prior to imaging;\footnote{Up to protein symmetries, which we discuss later.} this principle guides a number of applications in the field~\cite{frank2006three}.
Second, relative distances constrain the between orientations constrain them.

% Method overview (third figure).
Observing that, observing that
See figure .

We propose to learn a function---parametrized as a neural network---to predict the relative orientation between two projections based on their similarity.
This distance function then allows us to estimate, for any new projection dataset, the relative orientations between pairs of projections.
From these estimated distances, we then retrieve the orientation of each projection through an appropriate minimization scheme.
Our method is illustrated in \figref{schematic:method-overview}.

% Our method effectively (i) estimates a metric space, then (ii) embed/realize it in the space of 3D orientations, $\SO(3)$.

% From abstract
    Our approach consists of two steps: (i) the estimation of the relative distances between pairs of projections, and (ii) the recovery of the absolute orientation of each projection from these distances.
    In step (i), pairwise distances are estimated by a Siamese neural network trained on synthetic cryo-EM projections from resolved bio-structures.
    In step (ii), orientations are recovered by minimizing the difference between (a) the distances estimated from the projections and (b) the distances induced by the recovered orientations.

%to embed it on the space of 3D orientations, $\SO(3)$.
% In practice, our method consists of: (i) estimate the distance between pairs of projections, and (ii) recover

% Note; here was Distance Learning. paragraph

%If the distances are exact geodesic distances between orientations, then orientation recovery will find a perfect realization of the metric space, with zero error.
%In practice, distances estimated from projections will only be a proxy of true distances.
%Experiment \secref{results:orientation-recovery:sensitivity} shows that a lower distance estimation error translates to a lower orientation recovery error.

% Note; here was Orientation Recovery. paragraph

%\subsection{Unit Quaternions and the Geodesic Distance}
%\subsection{Parameterization of orientations}
\subsection{Representation of orientations}\label{sec:method:orientation-representation}
%\subsection{Orientations and distances}\label{sec:method:orientation-representation}
%\subsection{Distance between orientations}
%\mdeff{Laurène: Which word is better: Representation or parameterization? Or something else, but we need consistency.}
%\lau{Both are fine. Personally, I use parametrization in my thesis, but we can settle on the other one.}
% representation of orientations, parameterization of rotation matrices

%The projection $\mathbf{P}_{\bth}$ in \eqnref{imaging-model} represents an integration through the 3D particle made by the beam of electrons.
The orientation of the 3D particle with respect to the microscope's detector plane is a rotation relative to a reference orientation (\figref{imaging-geometry}).
The group of all 3D rotations under composition is identified with $\SO(3)$, the group of $3 \times 3$ orthogonal matrices with determinant~1 under matrix multiplication.
% non-commutative/non-Abelian\footnote{order of rotations matter}
A rotation matrix $\mathbf{R}_{\bth} \in \SO(3)$ can be decomposed as a product of $\binom{3}{2}=3$ independent rotations, for example as $\mathbf{R}_{\bth} = \mathbf{R}_{\theta_3} \mathbf{R}_{\theta_2} \mathbf{R}_{\theta_1}$, where $\bth = (\theta_3,\theta_2,\theta_1) \in [0,2\pi[ \, \times \, [0,\pi] \times [0,2\pi[$ are the (extrinsic and proper) Euler angles in the $ZYZ$ convention (the most widely used parameterization in cryo-EM)~\cite{sorzano2014interchanging}.
%As illustrated in \figref{imaging-geometry}, $\mathbf{R}_{\theta_1}$ is a rotation about the $Z$-axis by the angle $\theta_1$, $\mathbf{R}_{\theta_2}$ is a rotation about the $Y$-axis by the azimuthal (or tilt) angle $\theta_2$, and $\mathbf{R}_{\theta_3}$ is a second rotation about the $Z$-axis by the in-plane angle $\theta_3$.
%The Euler angles $\bth$ and the rotation matrix $\mathbf{R}_\bth$ are equivalent representations of the orientation of a projection.

While Euler angles are a concise representation ($3$ numbers for $3$ degrees of freedom) of orientation, they suffer from a topological constraint (there is no covering map from the $3$-torus to $\SO(3)$) which manifests itself in the \textit{gimbal lock} (the loss of one degree of freedom when $\theta_2=0$). %~\cite{koks2006explorations}
That makes their optimization by gradient descent problematic.
The optimization of rotation matrices, which are made of $9$ numbers, would require computationally costly constraints (orthogonality and determinant~1) to reduce the number of degrees of freedom to $3$.
Moreover, the distance between orientations cannot be directly computed from Euler angles and is costly (30 multiplications) to compute from rotation matrices~\cite{huynh2009metrics}.
We solve both problems by representing orientations with unit quaternions.

%\paragraph{Quaternions.}
Quaternions $q \in \mathbb{H}$ are an extension of complex numbers\footnote{The algebra $\mathbb{H}$ is similar to the algebra of complex numbers $\mathbb{C}$, with the exception of multiplication being non-commutative.} of the form $q = a + b\boldsymbol{i} + c\boldsymbol{j} + d\boldsymbol{k}$ where $a,b,c,d \in \R$.
Unit quaternions $q \in \mathbb{S}^3$, where $\mathbb{S}^3 = \big\{ q \in \mathbb{H}: \lvert q \rvert = 1 \big\}$ is the 3-sphere with the additional group structure inherited from quaternion multiplication, concisely and elegantly represent a rotation of angle $\theta$ about axis $(x_1, x_2, x_3)$ as $q = \cos(\theta/2) + x_1 \sin(\theta/2) \boldsymbol{i} + x_2 \sin(\theta/2) \boldsymbol{j} + x_3 \sin(\theta/2) \boldsymbol{k}$.
They parameterize rotation matrices as
\begin{equation*}
    \mathbf{R}_q =
    \begin{pmatrix}
        a^2+b^2-c^2-d^2 & 2bc-2ad & 2bd+2ac \\
        2bc+2ad & a^2-b^2+c^2-d^2 & 2cd-2ab \\
        2bd-2ac & 2cd+2ab & a^2-b^2-c^2+d^2
    \end{pmatrix}.
\end{equation*}
%Composition of rotations is given by multiplications of quaternions.
Note that $\mathbb{S}^3 \rightarrow \SO(3)$ is a two-to-one mapping (a double cover) as $q$ and $-q$ represent the same orientation, i.e., antipodal points of $\mathbb{S}^3$ are identified.
%\paragraph{Distances.}
Unlike Euler angles, $\mathbb{S}^3$ is isomorphic to the universal cover of $\SO(3)$.
Hence, the distance between two orientations, i.e., the length of the geodesic between them on $\SO(3)$, is given by
\begin{equation}
    \begin{aligned}
        d_q &: \mathbb{S}^3 \times \mathbb{S}^3 \rightarrow [0,\pi], \\
        d_q(q_i, q_j) &= 2 \arccos \left( \left| \langle q_i, q_j \rangle \right| \right),
    \label{eqn:distance:orientations}
    \end{aligned}
\end{equation}
where $\langle \cdot, \cdot \rangle$ is the inner product between two quaternions and the absolute value $\left| \cdot \right|$ ensures that $d_q(q_i, q_j) = d_q(q_i, -q_j)$.
The distance $d_q(q_i, q_j)$ corresponds to the magnitude (angle) of the rotation $\mathbf{R}_*$ such that $\mathbf{R}_{q_i} = \mathbf{R}_* \mathbf{R}_{q_j}$~\cite{huynh2009metrics}.

\todo{As opposed projections are mirrored, we cannot resolve chirality.\footnote{An object is chiral if it cannot be superposed on its mirror image by any combination of translations or rotations.}
Global orientation is lost by projecting, and chirality is lost by integrating.
%\mdeff{Not only a rotation, but an integration through $z_3$. (As opposed projections are mirrored, we cannot resolve chirality. Projecting looses global orientation, integrating looses chirality.)}
That's why we train on half coverage.}

%For the sake of conciseness, we shall use the term ``with orientation~$q$'' to refer to 2D/3D objects considered in an imaging geometry parametrized by $q$.

% Euler angles:
% - Euler angles are not unique
% - gimbal lock problem
% - intuition behind changing the basis between two coordinate systems is not so clear
% - to compare if 2 different projs are close to each other in their projection directions, it does not suffice comparing their 2 sets of Euler angle sets
% - 3 unknown parameters
% - just used for representation of input and output format of rotation

% Affine transformations / rotation matrices:
% - intuitive
% - unique
% - 3x3 matrix, 9 parameters
% - used for angle alignment due to its transformation versatility (e.g. mirrors)
% simpler to compose
% avoid problem of gimbal lock
%Rotation matrix: Minor disadvantage: Multiplication of matrices is ~2 times slower than quaternions. Minor Advantage: Matrix-vector multiplication is ~2 times faster, and large. Huge disadvantage: Normalization! Ghram-Shmit is asymmetrical, which does not give a higher order accurate answer when doing differential equations. More sophisticated methods are very complex and expensive.

% Quaternions:
% - intuition behind changing the basis between two coordinate systems is not so clear
% - 4D vector, 4 unknown parameters
% - much less known in EM community, but they can be used to describe rotation
% - any time we need to perform rotation on camera position, the quaternion is translated into its corresponding rotation matrix and then it is applied to the coordinates of the central coordinate system
% - mirroring using quaternions - we lose any intuition about how two projection directions (mirrored and non-mirrored) are related
% more compact and numerically stable
%Unfortunately, the intuition behind changing the basis between two coordinate systems is not so clear and performing reflections (also known as mirrors, flips) using quaternions we lose any intuition about how two projection directions (mirrored and non-mirrored) are related.
%However, quaternions are more compact (4D vectors/4 scalars) and numerically stable compared to rotation matrices.
%The quaternions strike a nice balance of both, Euler angles and rotation matrices, being small and free from gimbal lock.

% Axis angles
% - used for visualizations
%Axis (angle = length of axis) Minor advantage: Small. Moderate disadvantage: Multiplication and applying to a vector is slow with trig. Moderate disadvantage: North-pole singularity at length = 2*pi, since all axis directions do nothing. More code (and debugging) to automatically rescale it when it gets near 2pi.

\subsection{Distance learning}\label{sec:method:distance-learning}
%\subsection{Metric learning}%\label{sec:method:distance-learning}
%\subsection{Estimating Relative Orientations from Projections}
%\subsection{Relative orientation estimation}
%\subsection{Relative orientation estimation from projections}

%We capitalize on the powerful function approximation capabilities of neural networks and on our ability to faithfully model the cryo-EM imaging process for the generation of training data.

Our goal is to train a function---parametrized as a neural network---to predict the relative orientation between two projections based on their similarity.

We require a function to estimate the distance between the orientation of two 3D particles from the projections.
Designing such a function is however intricate if not impossible, partly because the invariants are difficult to specify.
As data is plenty
We instead capitalize on the powerful function approximation capabilities of neural networks and the availability of training data
supported by
Hence, we rather opt to \textit{learning} this distance function by parameterizing it as a neural network.
\figref{schematic:distance-learning} illustrates the proposed learning paradigm.
For its training, we capitalize on (i) the public availability of numerous 3D atomic models\footnote{\url{https://www.ebi.ac.uk/pdbe/emdb}} and (ii) our ability to model the cryo-EM imaging procedure in order to generate realistic cryo-EM projection datasets.

\begin{figure}
    \centering
    \includegraphics[width=\linewidth]{schematic_distance_learning}
    \caption{%
        \textbf{Distance learning.}
        We are looking for a distance $d_p$ between projections that is an accurate estimator of the distance $d_q$ between their orientations.
        We propose to parameterize $d_p$ as a Siamese neural network (SiameseNN), trained on a synthetic dataset of $P \approx 10^3$ projections with associated orientation.
        %\todo{Figure: feature distance (not similarity) $d_f$ is \eqnref{distance:projections}. Loss function is \eqnref{distance-learning}. "Same Structure/Weights" $\rightarrow$ "Same NN $\G$ and shared weights $w$".}
        %\mdeff{Why the output "loss" arrow? To me, this system is closed.}
        %\mdeff{"Loss Function" could be called Optimization or Learning or Training.}
}\label{fig:schematic:distance-learning}
\end{figure}

From a training dataset ${\{ \mathbf{p}_{i}, q_i \}}_{i=1}^{P}$ made of $P$ projections $\p_i \in \R^{n_p}$ with associated orientation $q_i \in \mathbb{S}^3$, we learn the \textit{projection distance}
\begin{equation}
    \widehat{d}_p = \argmin_{d_p} \sum_{i,j} \left| d_p\big(\mathbf{p}_{i},\mathbf{p}_{j}\big) - d_q\big(q_i,q_j\big) \right|^2,
    \label{eqn:distance-learning}
\end{equation}
with $d_q$ defined in~\eqnref{distance:orientations}.
The distance $d_p$ is parameterized as the Siamese neural network (SiameseNN)~\cite{chopra2005learning}
\begin{equation*}
    d_p(\p_i, \p_j) = d_f(\G_w(\p_i), \G_w(\p_j)),
    %\label{eqn:distance:projections}
\end{equation*}
where $\G_w$ is a convolutional neural network with weights $w$ that is trained to extract the most relevant features $\f_i \in \R^{n_f}$ from a projection $\p_i$. SiameseNNs, also termed ``twin networks'', are commonly used in the field of deep metric learning to learn similarity functions~\cite{yi2014deep}.
The distance in the feature space $d_f$ is often taken to be the Euclidean distance $d_f(\f_i, \f_j) = \Vert \f_i - \f_j \Vert_2$.
To facilitate the learning of a distance that respects the elliptic geometry of $\mathbb{S}^3$, we set $d_f = d_q$.

Evaluating the sum over $\frac{P^2-P}{2}$ pairs in \eqnref{distance-learning} is computationally intractable for cryo-EM datasets with typically $P \approx 10^5$ projections.
In practice, we minimize \eqnref{distance-learning} by stochastic gradient descent (SGD) over small batches of pairs and update weights by back-propagation.
% error / objective value

\subsection{Orientation recovery}\label{sec:method:orientation-recovery}
%\subsection{Orientation recovery from relative orientations}

%\paragraph{Orientation Recovery.}
The task of recovering points based on their relative distances has been extensively studied in the literature, primarily within the frameworks of dimensionality reduction and data visualization~\cite{belkin2003laplacian,kruskal1978multidimensional, maaten2008visualizing, mcinnes2018umap,dokmanic2015euclidean}.
These methods aim at mapping high-dimensional data onto a lower-dimensional space in such a way that the structure of the metric space is preserved.
Formally, the recovery is achieved by minimizing an appropriate loss function; well-known examples include Laplacian eigenmaps, multi-dimensional scaling (MDS), Isomap, LLE, t-SNE, and UMAP.
In the Euclidean setting, the embedding of distance matrices (given by their eigenvectors) is especially well-described.
In particular, the framework of the Euclidean distance matrices (EDMs)~\cite{dokmanic2015euclidean} provides theoretical guarantees on the retrieval of points from collected distances.

In our case, as we shall shortly explain, we aim to perform the embedding on $\SO(3)$, the non-Euclidean space of 3D rotations. Hence, the loss function to minimize (\secref{method:orientation-recovery}) is non-convex. Furthermore, we are not aware of any theoretical characterization of its behavior when input distances are lacking or corrupted by perturbations. That being said, the locally-Euclidean behavior of the $\SO(3)$ space offers some hope on the feasibility of this minimization.
Indeed, despite the lack of theoretical guarantees, we are able to appropriately minimize our loss function using a gradient-based algorithm, as we experimentally demonstrate in \secref{results:orientation-recovery:exact}.

Equipped with a trained estimator $\widehat{d}_p$, we then recover the orientations from a set of projections $\big\{ \mathbf{p}_k \big\}_{k=1}^P$ through
\begin{equation}
    \big\{ \widehat{q}_k \big\}_{k=1}^P = \argmin_{\{q_k \in \mathbb{S}^3\}} \sum_{i,j} \left| \widehat{d}_p \left( \p_i, \p_j \right) - d_q\left(q_i,q_j\right) \right|^2.
    \label{eqn:orientation-recovery}
\end{equation}
Note that the sole difference with~\eqnref{distance-learning} is that the minimization is performed over the orientations $q$ rather than the distance $d_p$.
Here again, the sum in \eqnref{orientation-recovery} is sampled for computational reasons.

A strategy, commonly employed by methods for sparse embedding like multidimensional scaling (MDS), Isomap, etc.~\cite{platt2004fast, 5438584,DBLP:journals/corr/abs-1811-10470} that amounts to build and embed a sparse distance graph.
More specifically, these algorithms take a matrix of distances and find vectors in a lower dimensional space that well match the inter-vector distances.
Hence, it is possible to determine the coordinates in a lower dimensional space only by knowing their pair-wise distances (at least in Euclidean space).
In our work, we are dealing with non-Euclidean space and we show that it is also possible to determine the coordinates in a lower dimensional space (quaternion space) knowing only the learned pair-wise geodesic distances between two projections.
%\lau{@Mdeff: Can you please fill here? Thanks.}
In practice, \eqnref{orientation-recovery} is also minimized by mini-batch SGD.
%We experimentally demonstrate in \secref{results:orientation-recovery:exact} that both approximations don't affect recovery performance.
