\section{Discussion}

\mdeff{Future work: fix or downplay the influence of larger distances in orientation recovery.}

\lau{Future work: extensive testing with other proteins, motivated by several results.}

\lau{Let's work on this once we have agreed on a final version of the experiments/results.} In this work, we explored the use of distance learning between two 2D projections of the 3D protein structure for the inference of the angles at which these projections were imaged from.
We present two entangled novel methods, distance learning and orientation recovery.

The results obtained so far provide key insights on the viability of the proposed methods.
They guarantee that each component of this new paradigm can handle the task it is designed for, and they provide some early indications on the robustness of the overall scheme.
%\lau{To update with latest results.}
After running the full machine learning pipeline and evaluating the performance on the synthetic data, we are able to reconstruct the asymmetric protein (\texttt{5j0n}) with the error of $0.1629$ rad and the symmetric protein (\texttt{5a1a}) with the error of $0.1871$ rad.
In addition, using these methods we are able to reconstruct the expected biological properties of the structure of these proteins.
%Unfortunately, we have not yet had the time to run an end-to-end pipeline and evaluate the performance of~\eqnref{orientation-recovery} when the trained SiameseNN is used as learned $\widehat{d_p}$; this is very clearly the next line of research.
%\mdeff{Don't start with the results, but the beauty of the method.}

We demonstrated that the method and overall idea works. Future research should focus on improving distance estimation, through a larger SiameseNN trained on more data, which will automatically improve orientation recovery (\secref{results:distance-estimation:sensitivity}) and reconstruction (\secref{results:orientation-recovery:reconstruction}).
More data = more proteins, more augmentation (noise, PSF).
We observe that the pipeline works, but for the state-of-the-art reconstruction we need a better distance estimation.
However, the method developed is promising since the learned distance is robust to perturbations.
We observe that the orientation recovery and distance estimation are interconnected, i.e., if one works the other one will work.

% \lau{Ok I keep the following paragraph here, but depending on the results, we could remove it.}
% \mdeff{Distance learning should stay the main source of improvement.}
% First, the results underline the importance of learning an accurate proxy distance $\widehat{d_p}$.
% In this regard, we could improve the performance of the SiameseNN in several ways, for instance by further tuning the architecture of its twin CNNs as well as the distance metric between the two CNN outputs. We could parametrize the function $d_f$, which compares the similarity of the features outputs (see \figref{schematic:distance-learning}), as a feed-forward neural network instead of the current Euclidean distance, and learn its weights as well.

Despite being able to infer the orientation of 2D projections of the 3D protein structure, new cryo-EM measurements of the proteins unseen from the network could enable a more direct validation of the results.
To achieve this, we shall rely on our powerful, expressive forward model of the cryo-EM procedure to generate realistic projections from thousands of atomic models in the PDB database.
In this regard, an interesting aspect of our methods for the present application is that they intrinsically predict the \textit{relationship} between objects.
Hence, a well-trained distance learning method could be relatively robust to the change of volumes.
In the same line of thought, our distance learning method will likely benefit from the profound structural similarity shared by proteins---after all, they all derived from just the same 21 amino acids.
Further down the line, an extended training will allow us to test different options for the challenging handling of proteins with multiple conformational states.
These methods have a potential to advance the understanding of the protein structure given only the set of the projections.

\todo{Still a proof-of-concept.
Method TODO:
\begin{itemize}
    \item Angle alignment didn't always work. Even when $L_\text{OR}$ was low (examples?). We might miss a transformation in \eqnref{orientation-recovery-error}.
    \item We needed to align with \eqnref{orientation-recovery-error} before reconstructing with ASTRA\@. Why?
    \item More data and larger/more powerful SiameseNN for better distance approximation. More data to close the overfitting gap (e.g., \texttt{5a1a}), more powerful to push $L_\text{DE}$ done.
\end{itemize}
Future work:
\begin{itemize}
    \item Sensitivity / invariance to PSF.
    \item Train on much larger and diverse projections (proteins, augmentation (noise levels, PSFs).
    \item Real (not synthetic) data.
    \item Proper reconstruction (beyond our simple ASTRA demo).
    \item Full pipeline for unseen proteins.
\end{itemize}
}

%We employ the definition of a correct estimation: the estimation must be within 10\degree~(0.174 rad) of true orientations for noiseless data and within 25\degree~(0.436 rad) for noisy data.\mdeff{Do we use this definition?}

% ABOUT SIAMESE_NN

%The success of the SiameseNN as a faithful predictor of relative orientations eventually relies on our capacity to generate a synthetic training dataset whose data distribution is diverse enough to cover that of unseen projection datasets.
%The objective is for the SiameseNN to be able to handle projections acquired in all sorts of imaging conditions and originating from 3D volumes it has never been trained on.

%We shall create such comprehensive training dataset by capitalizing on two favorable conditions.
%First, there exists a large publicly-available database of deposited atomic models of proteins, which gives us access to thousands of different 3D ground truths.
%Then, we shall take advantage of our ability to model the cryo-EM imaging procedure to generate, from these ground truths, a synthetic dataset that contains a massive amount of realistic projections whose orientations are, by definition, all known.

% Put somewhere?
%The method capitalizes on the powerful learning capabilities of neural networks, yet still fundamentally relies on our ability to faithfully model the cryo-EM imaging process for the generation of the training dataset.
