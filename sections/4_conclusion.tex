\section{Discussion}
\label{sec:discussion}

In this work, we explored the use of distance learning between pairs of 2D cryo-EM projections from a 3D protein structure to infer the unknown angles at which each projection was imaged from. Our two-step method relies on the training of a SiameseNN to estimate pairwise distances between unseen projections, followed by the recovery of the orientations from these distances through an appropriate minimization scheme. 

The benefit of this approach, at least in theory, is that it would permit to estimate the unknown orientations in single-particle cryo-EM directly from the acquired dataset, \ie, without the need for intermediate reconstruction procedure or initial volume estimate; this has obvious useful implications in the field.  

At the current stage of development, the method has been evaluated on synthetic datasets for two different proteins. The results provide key insights on the viability of the proposed scheme. First, they demonstrate that the distance learned by the SiameseNN is robust to increasing levels of noise on the projections---an important feat in cryo-EM. They also guarantee that accurate distance estimation leads to a correct recovery of the orientations, while providing some early indications that the quality of this recovery depends on the precision of the estimated distances. We ran the ``full pipeline'' on the two synthetic data and were able to recover the projection orientations with an error of about \todo{$0.16-0.19$ rad}. 

While the method is not yet at the stage where it can be routinely deployed in practice, we believe that a series of developments could help it become a relevant contributor for single-particle cryo-EM reconstruction.\footnote{\lau{Note that the present project will not be further continued by its authors due to other professional occupations. Hence, any interested ... is strongly encouraged to ...}} 

As previously discussed, the results underline the importance of learning an accurate distance $\widehat{d_p}$. In this regard, the performance of the SiameseNN could be improved in several ways. The most obvious enhancement would be to train the Siamese on a more exhaustive and diverse cryo-EM dataset. Indeed, the success of the SiameseNN as a faithful predictor of relative orientations eventually relies on our capacity to generate a synthetic training dataset whose data distribution is diverse enough to cover that of unseen projection datasets. Such realistic cryo-EM projections could be generated by relying on a more expressive formulation of the cryo-EM and taking advantage of the thousands of atomic models available in the PDB database. In particular, a necessary extension will be to include the effects of the PSF when generating the training dataset and evaluate its impact on the prediction of the SiameseNN. 

Gains could also be obtained by further tuning the architecture of the SiameseNN's twin CNNs, as well as the distance metric between the two CNN outputs. For instance, one could parametrize the function $d_f$---which compares the similarity of the features outputs (see \figref{schematic:distance-learning})---as a feed-forward neural network instead of the current Euclidean distance, and learn its weights as well.

A set of additional technical developments could also further improve the performance of the pipeline.  \todo{\textit{Is all that follows still relevant to include?} Fix or downplay the influence of larger distances in orientation recovery. Angle alignment didn't always work, even when $L_\text{OR}$ was low (examples?). We might miss a transformation in \eqnref{orientation-recovery-error}. We needed to align with \eqnref{orientation-recovery-error} before reconstructing with ASTRA\@ -Why?.}
\todo{The Siamese might learn better by predicting the differences in direction $(\theta_2,\theta_1)$ and in-plane angle $\theta_3$ separately.}

A final phase of tests before deploying the method on real data will be to extensively test the full pipeline on ``unseen proteins'', \ie, proteins whose simulated projections have never been seen by the SiameseNN. Early experiments seem to indicate the feasibility of this enterprise (see Figure~\ref{fig:robustness-to-unseen-pipeline} in Appendix).
In this regard, an interesting aspect of our method is that the twin CNNs within the SiameseNN intrinsically predict the \textit{relationship} between objects. 
Consequently, a well-trained distance learning could be relatively robust to the ``mismatch'' of volumes within the training set.
In the same line of thought, our distance learning method is likely to benefit from the profound structural similarity shared by proteins---after all, they all derived from the same 21 building blocks.

Eventually, the performance of the method on real cryo-EM measurements will provide the real measure of its potential, the imaging conditions being notoriously challenging in single-particle cryo-EM. Further down the line and still in the real of the hypothetical, new approaches for the training of the SiameseNN able to handle the handling of proteins with multiple conformational states could be explored. 

\lau{On a final note,  by encouraging extension from other contributors?}

