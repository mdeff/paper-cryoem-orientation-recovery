\documentclass{article}
% ====================================
% PACKAGES
% ====================================
\usepackage{arxiv}
\usepackage[utf8]{inputenc} % allow utf-8 input
\usepackage[T1]{fontenc}    % use 8-bit T1 fonts
\usepackage{microtype}      % microtypography
\usepackage{amsfonts}
\usepackage{amsmath}
\usepackage{bm}
\usepackage{hyperref}
\usepackage{graphicx}
\usepackage{mathtools}
\usepackage{enumitem}
\usepackage{setspace}
\usepackage{subcaption}
\usepackage{xcolor}
\usepackage{color}
\graphicspath{{figures/}}
\usepackage{upgreek}
\usepackage{cases}
\usepackage{arydshln}
\usepackage{wrapfig}
\usepackage{blkarray}
\usepackage{enumitem}
\usepackage{gensymb}  % for \degree

% ====================================
% COMMANDS
% ====================================

\newcommand{\figref}[1]{Figure~\ref{fig:#1}}
\newcommand{\tabref}[1]{Table~\ref{tab:#1}}
%\newcommand{\secref}[1]{Section~\ref{sec:#1}}
\newcommand{\secref}[1]{\S\ref{sec:#1}}
%\newcommand{\eqnref}[1]{(\ref{eqn:#1})}
\newcommand{\eqnref}[1]{\eqref{eqn:#1}}

% argmin
\newcommand\argmin[1]{
\underset{#1}{\arg\;\min}
}
% SO(3) space
\newcommand\SOThree[0]{
\mathrm{\mathbf{SO}}(3)
}
% Euler angles
\newcommand\bth[0]{{\boldsymbol{\theta}}}

% notes
\newcommand{\todo}[1]{{\color[rgb]{.6,.1,.6}{#1}}}
\newcommand{\banjac}[1]{{\color[rgb]{.3,.5,.9}{#1}}}
\newcommand{\donati}[1]{{\color[rgb]{.9,.5,.3}{#1}}}
\newcommand{\mdeff}[1]{{\color[rgb]{.8,.3,.2}{#1}}}


%\renewcommand{\headeright}{Technical Report}
%\renewcommand{\undertitle}{Technical Report}
\renewcommand{\headeright}{}
\renewcommand{\undertitle}{}

\author{
    \href{https://orcid.org/0000-0001-7373-4150}{\includegraphics[scale=0.06]{orcid}\hspace{1mm}}Jelena Banjac, \href{https://orcid.org/0000-0001-9834-7755}{\includegraphics[scale=0.06]{orcid}\hspace{1mm}}Laurène Donati, \href{https://orcid.org/0000-0000-0000-0000}{\includegraphics[scale=0.06]{orcid}\hspace{1mm}}Michaël Defferrard \\
    EPFL, Switzerland \\
    \texttt{\{jelena.banjac,laurene.donati,michael.defferrard\}@epfl.ch}
}

\date{}
\title{Learning to recover orientations from projections in single-particle cryo-EM}

%\title{Learning to recover the orientations of projections in single-particle cryo-EM}
%\title{Learning to recover the orientation of cryo-EM projections}
%\title{Recovering the orientation of cryo-EM projections from learned distance estimation}
%\title{Recovery of Orientations in SPA: Learning from Projections}
%\title{Learning to recover orientations from projections}
%\title{3D Poses Recovery in Single-Particle Cryo-EM from Learned Pairwise Projection Distances}
% keywords: learning, protein imaging/reconstruction (why/goal), (single-particle) Cryo-EM, SPA

%%%%%%%%%%%%%%%%%%%%%%%%%%%%%%%%%%%%%%%%%%%%%%%%%%%%%%%%%%%%%%%%%%%%

\begin{document}

\maketitle

\begin{abstract}
    A major challenge in single-particle cryo-electron microscopy (cryo-EM) is that the orientations adopted by the 3D particles prior to imaging are unknown; yet, their knowledge is essential for high-resolution reconstruction.
    We present a method to recover these orientations directly from the acquired set of 2D projections. Our approach consists of two steps: (i) the estimation of the relative distances between numerous pairs of projections, and (ii) the recovery of the absolute orientation of each projection from these distances.
    % absolute orientations vs relative distances?
    For step (i), the pairwise distance estimator is a Siamese neural network trained on synthetic cryo-EM projections from resolved bio-structures.
    For step (ii), the orientations are recovered by minimizing an appropriate loss function.
    Experimental results show that \lau{put here the latest results situation. %the quality of the recovered orientations strongly depends on the quality of the estimated distances.
    While not yet up to state-of-the art ..., distance estimation is robust to perturbations (adapt).}
    Overall, the orientations recovered from simulated noisy projections are within \todo{10\degree} of the true orientations.
    Our code is available at \url{https://github.com/JelenaBanjac/protein-reconstruction}.
\end{abstract}


\section{Introduction}

In single-particle analysis (SPA), the 3D particles adopt a random orientation in the ice layer before being imaged with parallel electron beams. Hence, the projection geometry associated to each 2D projection is unknown. Yet, this knowledge is essential for tomographic reconstruction. To handle this, a popular approach used by most SPA software packages is to alternatively refine the 3D structure and the orientation estimation~\cite{penczek1994ribosome,Baker1996,Dempster1977,sigworth1998maximum,scheres2012bayesian}. Iterative refinement procedures are extremely powerful and have permitted the determination of numerous biological structures up to near-atomic resolution~\cite{kuhlbrandt2014resolution}. Unfortunately, the outcome of these methods is predicated on the quality of the initial reconstruction, or, equivalently, on the initial estimation of the orientations~\cite{sorzano2006optimization,henderson2012outcome}. Several methods have been designed to produce a first rough structure~\cite{singer2010detecting,wang2013orientation,greenberg2017common,punjani2017cryosparc,pragier2019common}, but this remains a notoriously arduous challenge in SPA.

In this work, we present the outline and the preliminary results of an ongoing research project for SPA that capitalizes on the powerful learning capabilities of neural networks, yet still fundamentally relies on our ability to faithfully model the cryo-EM imaging process (for the generation of the training dataset). As we shall shortly detail, its target is the design of a method that learns to estimate the unknown orientation associated to each projection in a SPA dataset without relying on any intermediate reconstruction procedure. The method is still at its proof-of-concept stage, and several interesting developmental steps lie ahead.

\section{Method}

\subsection{Outline of the Proposed Method}

Our approach relies on the well-known observation that the greater the similarity between two projections, the more likely they originated from two 3D particles that adopted close orientations in the ice layer prior to imaging\footnote{Up to some possible intrinsic symmetries of the objects, which are discussed later.}. This principle guides a number of applications in SPA, including that of projection matching~\cite{penczek1994ribosome}.

Taking this line of thought further, we train a function---parametrized as a neural network---to predict the relative orientation between two projections based on their similarity. To make such training possible, we capitalize on our ability to model the cryo-EM imaging procedure to generate a large, representative synthetic dataset using publicly available 3D atomic models.

Using this trained distance function, we can estimate the relative orientations between pairs of projections in any real dataset. Our postulate is that we can then recover, from these estimated relative orientations, the orientations themselves through an appropriate minimization scheme. This two-steps pipeline is illustrated in Figure~\ref{fig:overview-pipeline}.

\begin{figure}
    \centering
    \includegraphics[width=\textwidth]{pipeline-overview}
    \caption{Overview of the proposed two-steps method: 1) estimate the relative orientations between projection pairs through a learned distance $\widehat{d}_b$, and 2) recover the orientations from the estimated relative orientations. We denote a $p$th projection by $\mathbf{b}^p$ and its orientation by $q_p$. The geodesic distance between two orientations is denoted by $d_q$.}
    \label{fig:overview-pipeline}
\end{figure}

The task of recovering points based on their relative distances has been extensively studied in the literature, mostly within the framework of dimensionality reduction and primarily for the case of \textit{Euclidean} embedding spaces\footnote{An ``embedding space'' corresponds to the (often lower-dimensional) space in which data is embedded, \textit{i.e.}, mapped to in such a way that the relative distances between its points are preserved as much as possible.}~\cite{belkin2003laplacian,kruskal1978multidimensional, maaten2008visualizing, mcinnes2018umap,dokmanic2015euclidean} . In that respect, the short example given by Dokmanic \textit{et al.} in~\cite{dokmanic2015euclidean} efficiently illustrates the philosophy behind such methods (see the ``An Analogy'' box).

\setlength{\fboxsep}{1em}\noindent\fbox{\parbox{0.95\textwidth}{%
\textbf{An Analogy: Mapping the Position of Swiss Cities with a Train Timetable~\cite{dokmanic2015euclidean}} \\

\begin{wrapfigure}{r}{0.45\textwidth}
  \begin{center}
    \includegraphics[width=0.4\textwidth]{swissEDM}
  \end{center}
  \caption{\footnotesize Image adapted from~\cite{dokmanic2015euclidean}. The red signs indicate the correct city locations. The black dots denote the recovered city locations.}
\end{wrapfigure}

\small
In this toy problem, the authors aim at estimating the position of five cities on the Swiss map based not on the spatial distances between them, but on the time it takes to travel by train between them. Those time data (in minutes) are collected in the following timetable: \vspace{0.25cm}

{\footnotesize\begin{blockarray}{cccccc}
& \text{L} & \text{G} & \text{Z} & \text{N} & \text{B} \\
\begin{block}{l(ccccc)}
  \text{Lausanne}  & 0   & 33  & 128 & 40 & 66 \\
  \text{Geneva}    & 33  & 0   & 158 & 64 & 101 \\
  \text{Zürich}    & 128 & 158 & 0   & 88 & 56 \\
  \text{Neuchâtel} & 40  & 64  & 88  & 0  & 34 \\
  \text{Bern}     & 66  & 101 & 56  & 34 & 0 \\
\end{block}\end{blockarray}.} \\

Remarkably, even though these time data only roughly correlate with the physical distances between the cities, one can still obtain a remarkably good estimate of their positions on the Swiss map (up to some symmetries of the embedding space) using a multidimensional scaling algorithm.
}}\normalsize \\

This example, if rather simple, nevertheless underlines well the key ingredients of methods that aim at retrieving points from distances that may not be directly measurable:

\begin{enumerate}
    \item \textit{An appropriate proxy for the ``real'' distance}. In the above example, the proxy for the spatial distance between two cities is the time taken to travel by train between them. In our case, we shall consider the similarity between two projections to be a good proxy for their relative orientation.
    \item \textit{A sufficiently rich collection of proxy distance data}. In this example, these data are provided by the (complete) train timetable. In our approach, we shall estimate the relative orientations between numerous pairs of projections based on the aforementioned proxy distance.
    \item \textit{An efficient recovery scheme}. In~\cite{dokmanic2015euclidean}, the embedding space being Euclidean, the theoretical framework of the Euclidean distance matrices (EDMs) guarantees that one can retrieve the desired points from the collected distances. In our case, as we shall shortly explain, we aim to embed the estimated relative orientations on $\SOThree$, the space of 3D rotations. Unfortunately, the extension of the EDM theory to such manifold is all but straightforward.
\end{enumerate}

There is no simple way to ``handcraft'' a proxy distance that would robustly predict the similarity between two projections. Hence, we resort to \textit{learning} this distance function by parametrizing it as a neural network and capitalizing on 1) the public availability of large datasets of 3D atomic models\footnote{\texttt{https://www.ebi.ac.uk/pdbe/emdb}}, and 2) our ability to model the cryo-EM imaging process. This is the topic of Section~\ref{sec:estimating-relative-orientations}.

Equipped with this learned distance, the idea is then to apply the aforementioned two-steps method (see Figure~\ref{fig:overview-pipeline}) for any projection dataset. As we just mentioned, we cannot rely on the theoretical framework of EDMs since our embedding space is non-Euclidean. Despite this lack of theoretical guarantees, we are able to appropriately minimize our objective function using a gradient-based algorithm, as we experimentally demonstrate in Section~\ref{sec:orientation-recovery}.

As a preamble, we discuss the need for a representation of orientations in $\SOThree$ that relies on unit quaternions.

\subsection{Unit Quaternions and the Geodesic Distance}
\label{sec:quaternions}

As mentioned, our objective is to recover unknown 3D orientations by embedding their estimated relative distances on the $\SOThree$ space. As we shall explain in the next sections, this embedding requires the efficient computation of the relative distance between two rotations $\mathbf{R}_1, \mathbf{R}_2 \in\SOThree$, which corresponds to the rotation $\mathbf{R}_*\in\SOThree$ such that $\mathbf{R}_1=\mathbf{R}_*\mathbf{R}_2$.

It is standard in SPA to work with Euler angles to describe the orientation of a 3D object in the electron microscope. More precisely, one relies on the parametrization $\bth=(\theta_1,\theta_2,\theta_3)\in\Omega_\bth$, with $\Omega_\bth=[0;2\pi)\times [0;\pi] \times [0;2\pi)$, to encode the 3D rotation that relates the object coordinate system to the projection coordinate system.

Unfortunately, the relative distance between two rotations $\mathbf{R}(\bth_1)$, $\mathbf{R}(\bth_2)$, parametrized by Euler angles cannot be directly computed from $\bth_1$, $\bth_2$. It requires the computation of the rotation matrices, which is computationally inefficient\footnote{Another technical challenge with Euler angles is that they suffer from the so-called gimbal lock problem, which arises when $\theta_2=0$ and restricts the number of rotational degrees of freedom to one even though $\theta_1$ and $\theta_3$ have not yet been fixed~\cite{koks2006explorations}.}. Hence, we resort to a more convenient representation of 3D rotations that relies on unit quaternions.

The algebra of quaternions was introduced in the mid-nineteenth century by Hamilton~\cite{rosenfeld_history_1988}. A quaternion $q\in\mathbb{H}$ takes the form
\begin{equation}
    \label{eq:quaternion-definition}
    q =  a\boldsymbol{1} + b\boldsymbol{i} + c\boldsymbol{j} + d\boldsymbol{k},
\end{equation}
where $(a,b,c,d)\in\mathbb{R}^4$, and $\boldsymbol{1}$, $\boldsymbol{i}$, $\boldsymbol{j}$, and $\boldsymbol{k}$ are the fundamental quaternion units
\begin{equation}
    \label{eq:quaternion-units}
    \boldsymbol{1} = \begin{pmatrix} 1 & 0 \\ 0 & 1 \end{pmatrix}, \quad
    \boldsymbol{i} = \begin{pmatrix} i & 0 \\ 0 & -i \end{pmatrix}, \quad
    \boldsymbol{j} = \begin{pmatrix} 0 & 1 \\ -1 & 0 \end{pmatrix}, \quad
    \boldsymbol{k} = \begin{pmatrix} 0 & i \\ i & 0 \end{pmatrix},
\end{equation}
with $i$ the imaginary unit. Any quaternion $q$ can thus be represented by its set of coefficients $(a,b,c,d)\in\mathbb{R}^4$. The algebra $\mathbb{H}$ is similar to the algebra of complex numbers $\mathbb{C}$, with the exception of the multiplication operation being noncommutative.

In this work, we restrict our interest to unit quaternions $q\in\mathbb{U}$, with  $\mathbb{U}=\big\{q\in\mathbb{H} \; \, | \; \,\lvert q \rvert =1\big\}$, which identify the $\mathbb{S}^3$ hypersphere in  $\mathbb{R}^4$. Unit quaternions concisely and elegantly represent the elements of the $\SOThree$ group. More precisely, a unit quaternion $q\in\mathbb{U}$ parametrizes a rotation $\mathbf{R}\in\SOThree$ through
\begin{equation}
    \mathbf{R}(q) =\begin{pmatrix}
    a^2+b^2-c^2-d^2 & 2bc-ad & 2bd+2ac  \\
    2bc+2ad & a^2-b^2+c^2d^2 & 2cd-2ab \\
    2bd-2ac & 2cd+2ab & a^2-b^2-c^2+d^2
    \end{pmatrix}.
    \label{eq:quaternion-rotation-matrix}
\end{equation}

The geodesic distance $d_q:\mathbb{U}\times\mathbb{U}\rightarrow [0,\pi]$ between two unit quaternions $q_i, q_j\in\mathbb{H}$ is then defined as
\begin{equation}
    \label{eq:geodesic distance}
    d_q(q_i,q_j)=2\arccos\big(|\langle q_i, q_j \rangle|\big),
\end{equation}
with the inner product between quaternions given by
\begin{equation}
    \label{eq:inner-product-quaternions}
    \langle q_i, q_j \rangle = a_ia_j+b_ib_j+c_ic_j+d_id_j.
\end{equation}
The distance~\eqref{eq:geodesic distance} is the shortest distance between $q_i$ and $q_j$ on the surface of $\mathbb{S}^3$.

As $\mathbb{S}^3$ is isomorphic to the universal cover of $\SOThree$, the geodesic distance corresponds to the magnitude of the relative orientation $\mathbf{R}_*$ between $\mathbf{R}(q_i)$ and $\mathbf{R}(q_j)$ in $\SOThree$~\cite{huynh2009metrics}. In other words, the relative distance between two rotations encoded by unit quaternions can be efficiently computed from the unit quaternions themselves through~\eqref{eq:geodesic distance}, which is of key practical importance for this work.

For the sake of conciseness, we shall use the term ``with orientation~$q$'' to refer to 2D/3D objects considered in an imaging geometry parametrized by $q$.

\subsection{Estimating Relative Orientations from Projections}
\label{sec:estimating-relative-orientations}

Equipped with the geodesic distance $d_q$, our goal is now to find a ``projection distance'' $d_b$ that is a good predictor of $d_q$. Before discussing the different options, we briefly describe the synthetic datasets used in this work.

\subsubsection{Experimental Dataset}
\label{subsec:datasets}

We consider two proteins as ground truths: the $\beta$-galactosidase, a protein with a dihedral (D2) symmetry, and the lambda excision HJ intermediate (HJI), an asymmetric protein. Their deposited PDB atomic models are 5a5a ~\cite{bartesaghi2015betagal} and 5j0n~\cite{laxmikanthan2016structure}, respectively. For each atomic model, we generate the ground truth by fitting a 5\AA\ density map in Chimera~\cite{pettersen2004ucsf}. We thus obtain a volume of size $(117\times 117\times 117)$ for the $\beta$-galactosidase, and a volume of size $(275\times 275\times 275)$ for the HJI.

From these ground truths, we generate $5,000$ synthetic projections of size $(117\times 117)$ and $(275\times 275)$, respectively, using the ASTRA projector~\cite{van2015astra}. The projection orientations are sampled from a uniform distribution over half the $\SOThree$ space, which suffices to generate all the possible projections of a volume. For the sake of simplicity, the projections are currently kept unblurred and noiseless. Whenever training neural networks, we split the datasets into a distinct training set (50\%), validation set (22\%), and testing set (33\%), to ensure that the results can generalize to unseen projections. The complete pipeline is implemented in Tensorflow~\cite{abadi2016tensorflow}.

\subsubsection{Baseline Test with the Euclidean Distance}

As a baseline, we first evaluate the suitability of the Euclidean distance as a projection distance $d_b$ to predict $d_q$. For the two aforementioned datasets, we randomly select $1,000$ pairs of projections. For each pair, we compute the Euclidean distance between the projections $d_b(\mathbf{b}^i,\mathbf{b}^j)=\lVert\mathbf{b}^i-\mathbf{b}^j\rVert_2$ and their relative orientation $d_p(q_i,q_j)$ through~\eqref{eq:geodesic distance}. We then report the $(d_q,d_b)$ relationship for all pairs in Figure~\ref{fig:euclidean-not-robust}, for both the asymmetric protein (left) and the symmetric one (right).

Two principal observations can be made from this experiment. First, as suspected, the Euclidean distance between projections fails to be a consistent predictor of their relative orientation distance, even in the simple imaging conditions considered here (no noise and no effect of the PSF). In particular, the larger the relative distance $d_q$, the poorer the predictive ability of the Euclidean distance as $d_b$. The other interesting observation is that the intrinsic symmetry of the $\beta$-galactosidase protein (5a1a) appears in its $(d_q,d_b)$ plot.

\begin{figure}
    \centering
    \includegraphics[width=\textwidth]{EuclideanDistance_NonRobust}
    \caption{Plotting the Euclidean distance between two projections versus their actual relative orientation (measured by the geodesic distance) for \textbf{(left)} the asymmetric protein (5j0n) dataset, and \textbf{(right)} the symmetric protein (5a1a) dataset. }
    \label{fig:euclidean-not-robust}
\end{figure}

\subsubsection{Learning $\widehat{d}_b$ with a Siamese Neural Network}

As previously discussed, we make the choice to \textit{learn} a good approximation $\widehat{d}$ on a synthetic training dataset $\big\{ \mathbf{b}^{*p}, q^*_p\big\}_{p=1}^{N_t}$ through
\begin{equation}
    \label{eq:metric-learning-siamese}
    \widehat{d}_b=\argmin{d_b}\sum_{i,j} \big|d_b\big(\mathbf{b}^{*i},\mathbf{b}^{*j}\big) - d_q\big(q^*_i,q^*_j\big) \big|^2,
\end{equation}
with $d_q$ defined in~\eqref{eq:geodesic distance}, and where $N_t$ indicates the number of projection-orientation pairs in the training dataset. More precisely, we parametrize the distance function $d_b$ in~\eqref{eq:metric-learning-siamese} as a Siamese neural network (SiameseNN)~\cite{chopra2005learning}, and resort to learning its weights $w$, as illustrated in Figure~\ref{fig:siamese-schematic}.

SiameseNNs, also termed ``twin networks'', are commonly used in the field of deep metric learning to learn similarity functions~\cite{yi2014deep}. They are usually constituted of two sister neural networks that work in tandem and share the exact same architecture and weights.  Their role (once trained) is to extract the projection features that are the most relevant to predict the relative orientation between two projections. The weights $w$ of the two sister networks are progressively learned by 1) comparing the difference of their projection feature vectors to the magnitude of the corresponding relative orientations, and 2) back-propagating this error (via the derivative chain rule) to the weights.

\begin{figure}
    \centering
    \includegraphics[width=\textwidth]{siameseNN-schematic}
    \caption{Training a Siamese neural network (SiameseNN) to become a faithful predictor of the relative orientation between two input projections. In other words, we train the SiameseNN to serve as a projection distance $\widehat{d}_b$ that correctly approximates the orientation distance $d_q$. The training is performed with a synthetic dataset that contains thousands of projections with their associated orientation.}
    \label{fig:siamese-schematic}
\end{figure}

\subsubsection{Generating a Proper Training Dataset for the SiameseNN}
\label{sec:training-siamese}

The success of the SiameseNN as a faithful predictor of relative orientations eventually relies on our capacity to generate a synthetic training dataset that is both large and representative of SPA measurements. In other words, we need to create a training set whose data distribution is diverse enough to cover that of unseen projection datasets. The objective is for the SiameseNN to be able to handle projections acquired in all sorts of imaging conditions and originating from 3D volumes it has never been trained on.

We shall create such comprehensive training dataset by capitalizing on two favorable conditions. First, there exists a large publicly-available database of deposited atomic models of proteins, which gives us access to thousands of different 3D ground truths. Then, we shall take advantage of our ability to model the cryo-EM imaging procedure to generate, from these ground truths, a synthetic dataset that contains a massive amount of realistic projections whose orientations are, by definition, all known.

Note that an interesting aspect of SiameseNNs for the present application is that they intrinsically predict the \textit{relationship} between objects. Hence, a well-trained SiameseNN could be relatively robust to the change of volumes. In the same line of thought, our SiameseNN will likely benefit from the profound structural similarity shared by proteins---after all, they all derived from just the same 21 amino acids.

\begin{figure}
    \centering
    \begin{subfigure}[t]{0.4\textwidth}
        \includegraphics[width=0.98\textwidth]{TrainingSiamese_LossAssymetric}
        \caption{Training losses of the SiameseNN on the asymmetric protein (5j0n) training and validation datasets.}
        \label{fig:losses-siamese-assym}
    \end{subfigure} \quad \quad
    \begin{subfigure}[t]{0.4\textwidth}
        \includegraphics[width=0.98\textwidth]{TrainingSiamese_LossSymetric}
        \caption{Training losses of the SiameseNN on the symmetric protein (5a1a) training and validation datasets.}
        \label{fig:losses-siamese-sym}
    \end{subfigure} \vspace{0.45cm}
    \begin{subfigure}[t]{0.4\textwidth}
        \includegraphics[width=0.98\textwidth]{TrainingSiamese_PlotAssymetric}
        \caption{Relative orientations predicted by the trained SiameseNN from projections in the asymmetric protein (5j0n) testing dataset. }
        \label{fig:learned-distance-siamese}
    \end{subfigure} \vspace{0.35cm}
    \caption{Training results for the SiameseNN.}
    \label{fig:losses-siamese}
\end{figure}

\subsubsection{Preliminary Training Results}

We present here a preliminary evaluation of the ability of SiameseNNs to learn a projection distance $\widehat{d}_b$ that correctly approximates the orientation distance $d_q$.

SiameseNNs come with a variety of more or less powerful architectures. At the current stage of development, we work with a simple one. Our SiameseNN is composed of two convolutional neural networks (CNNs) with shared weights. Their output features vectors are compared through an Eulidean distance, \textit{i.e.}, $F(\mathbf{x},\mathbf{y})=\lVert \mathbf{x}-\mathbf{y}\rVert_2$ in Figure~\ref{fig:siamese-schematic}. The detailed architecture of this SiameseNN is given in Figure~\ref{fig:app-SiameseNN-architecture} in Appendix X.

For each protein, we train the SiameseNN on its training dataset for 250 epochs ($\sim$10 hours) using an Adam optimizer~\cite{kingma2014adam}, a learning rate of $10^{-3}$, and a batch size of 256 projections. The evolution of the training and validation losses are presented in Figure~\ref{fig:losses-siamese-assym} for the asymmetric protein (5j0n), and in Figure~\ref{fig:losses-siamese-sym} for the symmetric one (5a1a). The results demonstrate that the SiameseNN succeeds at learning a proxy distance for the asymmetric protein dataset, as convergence is reached in about 50 epochs ($\sim$ 2 hours).

However, the current SiameseNN architecture fails at learning the distance for the dataset 5a1a, which is very likely due to the symmetry of the $\beta$-galactosidase protein. Indeed, its synthetic dataset contains pairs of projections that share the same $d_b$, yet differ in their $d_q$. This simply advocates for the restriction to non-overlapping areas on $\SOThree$ when sampling the orientations used to generate the SiameseNN training dataset. The latter would then only contain projection pairs with a linear $(d_q,d_b)$ relationship, which should ensure a successful training of the network. For the rest of the experiments, we use the asymmetric protein (5j0n) dataset.

We then feed to the trained SiameseNN $1,000$ pairs of projections randomly selected from the 5j0n testing dataset, and report the $(d_q,\widehat{d}_b)$ relationship of each pair in Figure~\ref{fig:learned-distance-siamese}. These results confirm that, for this protein at least, the SiameseNN is able to predict the orientation distance $d_q$ using only the projections as inputs. Moreover, it clearly outperforms the Euclidean distance at doing so. These preliminary results are encouraging, as much has yet to be gained from improving upon the rather primitive SiameseNN architecture we currently use.

\subsection{Orientation Recovery}
\label{sec:orientation-recovery}

Equipped with an appropriately learned $\widehat{d}_b$, the objective is then to recover the unknown unit quaternions $\big\{q_p\big\}_{p=1}^P$ associated to the projections $\big\{\mathbf{b}^p\big\}_{p=1}^P$ in any given dataset.

\subsubsection{Minimization Scheme}

We propose to start this process by computing of a great number of pairwise projection distances $\big\{\widehat{d}_b\big(\mathbf{b}^i,\mathbf{b}^j\big)\big\}_{i,j=1}^{P}$ through $\widehat{d}_b$. Then, our postulate is that we can recover the orientations from theses distances by solving
\begin{equation}
    \label{eq:global-min-problem}
    \big\{\widehat{q}_p\big\}_{p=1}^P=\argmin{q_i\in\mathbb{U}}\sum_{i,j} \big|\widehat{d}_b\big(\mathbf{b}^i,\mathbf{b}^j\big) - d_q\big(q_i,q_j\big) \big|^2,
\end{equation}
as is illustrated in Figure~\ref{fig:overview-pipeline}.

In practice, one cannot possibly evaluate~\eqref{eq:global-min-problem} for every pair of orientations $\big\{q_i,q_j\big\}_{i,j=1}^P$ given the extremely large size of SPA datasets, with $P$ typically in the order of dozens of thousands. Hence, we need to partially sample the projection dataset. We experimentally demonstrate in Section~\ref{subsec:5-6-3-sanity-check} that this does not affect the performance of our recovery scheme.

As previously discussed, we are not yet aware of any guarantee of convergence for~\eqref{eq:global-min-problem}. Similarly, we do not know of any theoretical characterization of the behaviour of~\eqref{eq:global-min-problem} in ill-posed conditions, such as when pairwise distances are misestimated, for instance. Hence, we rely for now on experimental demonstrations to 1) ensure feasibility, and 2) indicate where efforts need to be invested.

\begin{figure}
    \centering
    \begin{subfigure}[b]{0.48\textwidth}
        \includegraphics[width=\textwidth]{fig_perfectdistances_loss-symmetric}
        \caption{}
    \end{subfigure} \quad
    \begin{subfigure}[b]{0.48\textwidth}
    \centering
        \includegraphics[width=0.8\textwidth]{fig-perfectdistances-coverage-symmetric}
        \caption{}
    \end{subfigure}
    \caption{Results of the orientation recovery scheme when using the perfect orientation distances for the asymmetric protein (5j0n). \textbf{(a)} Evolution of the loss of~\eqref{eq:global-min-problem} during minimization. \textbf{(b)} Coverage of $\SOThree$ after the orientation recovery from the perfect relative distances. }
    \label{fig:minim-loss-perfect-distances}
\end{figure}

\subsubsection{Feasibility Check: Recovery from the Exact Relative Distances}
\label{subsec:5-6-3-sanity-check}

Our first investigation is to verify that it is at all possible to recover the orientations through~\eqref{eq:global-min-problem} from their true relative distances (\textit{i.e.}, using $d_q$ and not a proxy $d_b$).

We use the $5,000$ projections from the asymmetric protein (5j0n) dataset. Out of the possible 25 mio possible pairs, we randomly select only $5,000$ of them and compute their geodesic distance through~\eqref{eq:geodesic distance}. We then minimize~\eqref{eq:global-min-problem} using the SGD Adam optimizer~\cite{kingma2014adam} for $30K$ steps ($\sim$1 hour) with a learning rate of $0.1$.

The results are given in Figure~\ref{fig:minim-loss-perfect-distances}. They confirm that it is possible to recover the orientations from their true relative distances, even though the embedding space is non-Euclidean. As previously discussed, this is not a straightforward result. The results also demonstrate that a large subsampling of the projection pairs does not affect the convergence of~\eqref{eq:global-min-problem}, which is in straight line with the observations made by numerous Euclidean-based dimensionality reduction works~\cite{belkin2003laplacian,kruskal1978multidimensional, maaten2008visualizing, mcinnes2018umap}.

\subsubsection{Robustness of Recovery to Additive Errors on the Relative Distances}
\label{subsec:5-6-4-robustness-to-errors}

We now go one step further and evaluate the behaviour of~\eqref{eq:global-min-problem} when the true relative distances are corrupted by  additive Gaussian noise.

The experimental conditions are the same than in the previous section, except that we add an error with increasing variance on the relative distances prior to the minimization. The results are presented in Figure~\ref{fig:recovery-noise-distances} (red curve).

\begin{figure}
    \centering
    \includegraphics[width=0.75\textwidth]{fig-robustness-asymmetric}
    \caption{Results of the recovery scheme (red curve) and the alignment procedure (blue curve) when an increasing amount of errors is added to the true relative distances.}
    \label{fig:recovery-noise-distances}
\end{figure}

Before discussing the results, we remark that one cannot really quantify the performance of~\eqref{eq:global-min-problem} through its loss. Unfortunately, it is also not appropriate to directly compute the error between the recovered orientations $\big\{\widehat{q}_p\big\}_{p=1}^P$ and the true ones $\big\{q_p\big\}_{p=1}^P$. The reason is that the recovery of orientation through~\eqref{eq:global-min-problem} is up to a global rotation, \textit{i.e.}, any global rotation of the set of recovered orientations is as valid as any other. This is not a problem for the ultimate application of our scheme, but it complicates the quantitative evaluation of its performance in synthetic experiments. We circumvent this problem by 1) aligning the true and recovered orientation sets, and 2) computing their distance after alignment. The alignment is performed by searching for the orthogonal matrix (with determinant $\pm$ 1) $\mathbf{T}\in\mathbb{R}^4$  that minimizes

\begin{equation}
    \label{eq:alignement}
    \widehat{\mathbf{T}}=\argmin{\mathbf{T}\in\mathbb{R}^4}\sum_{i,j} \big|d_q\big(q_i,q_j\big)- d_q\big(\mathbf{T}\widehat{q}_i,\mathbf{T}\widehat{q}_j\big)\big|^2.
\end{equation}

For all variances, the distance after alignment is reported in Figure~\ref{fig:recovery-noise-distances} (blue curve).

These results demonstrate that the performance of the minimization scheme~\eqref{eq:global-min-problem} linearly depends on the quality of the relative distances, which advocates for a proper and extensive training of the SiameseNN in the next stages of development. Another interesting output of Figure~\ref{fig:recovery-noise-distances} is that it indicates that the error of the orientation recovery behaves as a monotonic function of its loss. Hence, it suggests that the loss can be used as a good indicator of its performance, which has obvious practical implications for our future works on real data.

\section{Results}

\subsection{Datasets}\label{sec:results:data}

%\paragraph{Proteins.}
We consider two proteins: the $\beta$-galactosidase, a protein with a dihedral (D2) symmetry, and the lambda excision HJ intermediate (HJI), an asymmetric protein with local cyclic (C1) symmetriy. Both can be seen on \figref{pdb-proteins}.
Their deposited PDB atomic models are \texttt{5a1a}~\cite{bartesaghi2015betagal} and \texttt{5j0n}~\cite{laxmikanthan2016structure}, respectively.
For each atomic model, we generate the ground truth by fitting a 5\AA\ density map in Chimera~\cite{pettersen2004ucsf}.
We thus obtain a volume of $110 \times 155 \times 199$ voxels for the $\beta$-galactosidase, and a volume of $69 \times 57 \times 75$ voxels for the HJI.

\begin{figure}[ht!]
    \centering
    \begin{subfigure}[b]{0.45\textwidth}
        \includegraphics[height=5.5cm]{figures/5a1a_pdb.png}
        \caption{}
    \end{subfigure}
    %\hfill
    \begin{subfigure}[b]{0.45\textwidth}
    \centering
        \includegraphics[height=5.5cm]{figures/5j0n_pdb.png}
        \caption{}
    \end{subfigure}
    
    \caption{%
        The two proteins we are working with.
        \textbf{(a)} $\beta$-galactosidase (\texttt{5a1a})~\cite{5a1a_pdb}.
        \textbf{(b)} lambda excision HJ intermediate (HJI) (\texttt{5j0n})~\cite{5j0n_pdb}.
    }\label{fig:pdb-proteins}
\end{figure}


\paragraph{Projections.}
From these ground truths, we generate $5,000$ synthetic projections of size $275\times 275$ and $116\times 116$, respectively, using the ASTRA projector~\cite{van2015astra}.
Our projection generator supports two orientation samplings: (i) sampling the Euler angles $\bth=(\theta_1,\theta_2,\theta_3)$ uniformly, and (ii) sampling uniformly on $\SO(3)$.
Due to protein symmetries, orientations are sampled differently. For the entire \texttt{5j0n} complex is asymmetric~\cite{doi:10.1002/9780470514160.ch4}, thus it is sufficient to sample the \textit{half} of the $\mathbb{S}^2$ sphere, since the other half will have equivalent projections that are located symmetric to the center of this sphere. Conversely, the $\beta$-galactosidase has D2 symmetry, i.e. it is composed of four identical sub-units got with two rotations of magnitude 180\degree around the first axis followed by 180\degree rotation around second axis, as illustrated and explained in~\cite{symmetry_in_protein,symmetry,scipion-em-github, rcsb-symmetry-view, EmpereurMot2019GeometricDO}. Therefore, for this protein we restrict the sampling to the quarter of $\mathbb{S}^2$ sphere.
%\mdeff{We should explain why.}
\figref{different-projections} shows example projections.

\begin{figure}[ht!]
    \centering
    \begin{subfigure}[b]{0.3\textwidth}
        \includegraphics[height=4.8cm]{figures/5j0n_noise0}
        \caption{}
    \end{subfigure}
    \hfill
    \begin{subfigure}[b]{0.3\textwidth}
    \centering
        \includegraphics[height=4.8cm]{figures/5j0n_noise16}
        \caption{}
    \end{subfigure}
    \hfill
    \begin{subfigure}[b]{0.3\textwidth}
    \centering
        \includegraphics[height=4.8cm]{figures/5j0n_translated}
        \caption{}
    \end{subfigure}
    \caption{%
        An example projection being perturbed.
        \textbf{(a)} Unperturbed projection $\mathbf{P}_{\bth_i} \mathbf{x}$.
        \textbf{(b)} Perturbed projection $\mathbf{P}_{\bth_i} \mathbf{x} + \mathbf{n}$, with noise $\mathbf{n}$ sampled from a Gaussian distribution of mean 0 and variance 16.
        \textbf{(c)} Perturbed projection $\mathbf{S}_{\mathbf{t}} \mathbf{P}_{\bth_i} \mathbf{x}$, with translations $t_1$ and $t_2$ sampled from a triangular distribution with a lower limit of -20 pixels, an upper limit of 20, and a mode (i.e., peak) of 0.
    }\label{fig:different-projections}
\end{figure}

\paragraph{Perturbations.}
We consider the following perturbations to control the difficulty of orientation recovery: (i) additive white noise, (ii) translations, (iii) point-spread functions (PSF). Mathematical formulation of these three components is shown in the equation \eqnref{projection-eqn}.


\begin{table}[ht!]
    \centering
    \begin{tabular}{lrrr}
        \toprule
        Dataset & Number of projections $P$ (\%) & Maximum number of pairs $P^2$ & Used number of pairs \\
        \midrule
        Train & 2512 (50\%) & 6,312,656 & 63,126 (1\%) \\
        Validation & 1650 (33\%) & 2,722,500 & 27,225 (1\%) \\
        Test & 838 (17\%) & 701,406 & all (sampled per batch) \\
        \bottomrule
    \end{tabular}
    \caption{
        Split of $P=5000$ projections (for both \texttt{5j0n} and \texttt{5a1a}) in training, validation, and test sets.
    }\label{tab:dataset}
\end{table}

\paragraph{Distance learning.}
We use supervised learning, thus input-output pairs.
The input are two images and the output is their quaternion distance calculated from the ground truth orientations.
For neural network training, dataset is split into a distinct training, validation, and testing set, see \tabref{dataset}.
The total number of generated projections is $P = 5,000$. Therefore, the number of possible projection pairs is $P^2 = 25e6$. If we split $P^2$ into the training, validation, and testing sets, it would mean that some of the projections that appear in the training dataset projection pairs can appear in pairs of the other dataset. To ensure that the results can generalize to unseen data as well as to understand how the model would perform in a real-world scenario, we split the projections $P$ (and not $P^2$) into training, validation, and testing projection sets. With these three projections sets we can create disjoint projection pair datasets sets (column with $P^2$ values in \tabref{dataset}). In addition to this, we will not use all projection pairs of each dataset, only $1\%$ of the possible pairs (last column in the \tabref{dataset}) due to limitation of available resources for the training.
%\todo{Better explain why projections (and not pairs) must be separated in the various sets.}


\paragraph{Orientation recovery.}
The orientation recovery is solving \eqnref{orientation-recovery} and it was done in a stochastic setting, where the loss function varies over the batches.
Hence, the dataset used in this part is test set from \tabref{dataset}.
Orientation recovery is performed on projections unseen during distance learning.

\subsection{Evaluation}\label{sec:results:evaluation}

%\mdeff{Story: need to know how good we did (without reconstructing the protein).
%As orientations are up to a global rotation/mirror on $\SO(3) / \mathbb{S}^3$, best align recovered and true orientations before computing the mean recovery error.}

%\todo{Introduce the mean recovery error as a good and intuitive performance metric.}
%\todo{Figure that shows a typical convergence and mean orientation recovery error before and after alignment. We'll subsequently only report $E$ (the error after alignment).}

Before discussing the results, we remark that one cannot really quantify the performance of~\eqnref{orientation-recovery} through its loss nor visual judgement of the protein reconstruction.
Unfortunately, it is also not appropriate to directly compute the error between the recovered orientations ${\big\{\widehat{q_p}\big\}}_{p=1}^P$ and the true ones ${\big\{q_p\big\}}_{p=1}^P$.
The reason is that the recovery of orientation through~\eqnref{orientation-recovery} is up to a global rotation, \textit{i.e.}, any global rotation of the set of recovered orientations is as valid as any other, which is mathematically written as $d_q(q_i, q_j) = d_q(\T q_i , \T q_j) \; \forall \, \T \in \Or(4)$.
This is not a problem for the ultimate application of our scheme, but it complicates the quantitative evaluation of its performance in synthetic experiments.
We circumvent this problem by aligning the true and recovered orientation sets up to a global rotation and mirror on $\SO(3)$ space \textit{i.e.} objective is to minimize the distance difference between these two sets.
%\mdeff{Note (again?) that $\mathbb{S}^3 \subset \R^4$ is a double cover of $\SO(3)$.}

The goal of the alignment is to compute the \textit{mean orientation recovery error}
\begin{equation}
    E = \min_{\T \in \Or(4)} \frac{1}{P} \sum_{i=1}^P \big| d_q\left( q_i, \T \widehat{q_i} \right) \big|,
    \label{eqn:orientation-recovery-error}
\end{equation}
where $\Or(4) = \{\T \, | \det(\T) = \pm 1\}$ is the group of $4 \times 4$ orthogonal matrices that represent the symmetries of $\mathbb{S}^3$ (isometries of $\R^4$) of 4D Euclidean space (i.e., rotations and reflections).

We implement $\T$ as the product of the $\binom{4}{2}=6$ possible 2D rotations in 4D space and an optional reflection (also known as flip or mirror):
\begin{equation*}
    \T =
%    \begin{bmatrix}
%        m & 0 & 0 & 0 \\
%        0 & 1 & 0 & 0 \\
%        0 & 0 & 1 & 0 \\
%        0 & 0 & 0 & 1 \\
%    \end{bmatrix}
    \begin{bmatrix}
        m & \mathbf{0} \\
        \mathbf{0} & \mathbf{I} \\
    \end{bmatrix}
    \prod_{i < j \leq 4} \mathbf{R}_{x_i x_j}(\theta_{x_i x_j}),
%    \prod_{(i,j) \in \{(1,2), (1,3) \dots\}} \mathbf{R}_{x_i x_j}(\alpha_{x_i x_j})
%    \mathbf{R}_{x_1 x_2}(\alpha_{x_1 x_2}) \mathbf{R}_{x_1 x_3}(\alpha_{x_1 x_3}) \mathbf{R}_{x_1 x_4}(\alpha_{x_1 x_4}) \mathbf{R}_{x_2 x_3}(\alpha_{x_2 x_3}) \mathbf{R}_{x_2 x_4}(\alpha_{x_2 x_4}) \mathbf{R}_{x_3 x_4}(\alpha_{x_3 x_4}),
    \quad m \in \{-1,1\}, \; \theta_{x_i x_j} \in [0, 2\pi[,
\end{equation*}
where $m = \det(\T) = -1$ if $\T$ includes a reflection, and $\mathbf{R}_{x_i x_j}(\theta_{x_i x_j}) \in \mathbf{SO}(4)$ is a rotation by angle $\theta_{x_i x_j}$ on the plane spanned by the $x_i$ and $x_j$ axes.

In practice, we again sample the sum and minimize \eqnref{orientation-recovery-error} by gradient descent with the FTRL optimizer~\cite{mcmahan_ad_2013}.
%noauthor_tfkerasoptimizersftrl_nodate}.
Because $\Or(4)$ is disconnected, we optimize the 6 angles separately for $m = 1$ (proper rotations) and $m = -1$ (improper rotations).
%Unless stated otherwise
We run FTRL with a learning rate of $2.0$, a learning rate power of $-2.0$, and a batch size of 256; and report the lowest of 6 runs (3 per value of $m$) of 300 steps each.

%\todo{Takes too much space for its relevance.}
%\todo{Make sure $\mathbf{R}$ is introduced like this before.}

\figref{5j0n-aa-loss-perfect-distances} shows a successful convergence and mean orientation recovery error before and after alignment with the perfect distance $d_q$. 
There many different ways of evaluating the pipeline performance found in the field of pose estimation. Some of the evaluations include the following:
\begin{itemize}
\item Intersection over Union (IoU) of the object 3D cloud with a custom threshold classifying it as a good estimate or not (e.g. in the paper~\cite{10.1007/s11263-014-0733-5} the threshold score above 0.5 is considered good estimation).
\item Translation and rotation error between estimated 3D model and true 3D model with fixed thresholds (e.g. in the paper~\cite{shotton2013scene} they require the translation error to be below 5 cm and rotation error to be below 5\degree)
\item The average distance of all the points of the model from their transformed version, and if the error is less than the constant multiple of diameter of the 3D model, it is considered correctly evaluated (e.g. evaluation error is used in papers \cite{10.1007/978-3-642-37331-2_42, xiang2018posecnn})
\item Reprojection error that projects the estimated points onto the image and computes the pairwise distances in the image space, instead of computing distances in the 3D model space (e.g. used in paper~\cite{xiang2018posecnn})
\item The recovery error measured as Frobenius norm from estimated 3D model and true model, where 3D model is composed of 3D locations of important landmarks (e.g. elbow for human pose estimation)~\cite{wangni2018monocular}
\item Average Orientation Similarity (AOS) is the difference between the true and estimated model with a cosine similarity term~\cite{RedondoCabrera2016PoseEE}
\item Mean Angle Error (MAE) and Median Angle Error (MedError) evaluated and compared with other pose estimation error metrics in the paper~\cite{RedondoCabrera2016PoseEE}.
\end{itemize}
Since mean orientation error is used in the pose estimation tasks and it is considered reliable performance metric, we decided to use it as our performance measure. We employ the definition of a correct estimation: the estimation must be within 10\degree~(0.174 rad) of true orientations for noiseless data and within 25\degree~(0.436 rad) for noisy data.

%\mdeff{Why is it good? Intuitive sure. Laurène, can we say something more?}

%%%%%%%%%%%%%%%%%%%%%%%%%%%%%%%%%%%%%%%%%%%%%%%%%%%%%%%%%%%%%%%%%%%%%%%%%%%%%%%%%%%%%%%
\subsection{Orientation recovery}\label{sec:results:orientation-recovery}

%\mdeff{Story: good distance estimation = good orientation recovery.}
% \begin{algorithm}[H]
% \SetAlgoLined
% \KwResult{Write here the result }
%  initialization\;
%  \For{$steps \gets 1$ \textbf{to} $30000$}{
%   instructions\;
%   \eIf{condition}{
%   instructions1\;
%   instructions2\;
%   }{
%   instructions3\;
%   }
%  }
%  \caption{Orientation recovery algorithm}
% \end{algorithm}

%\subsubsection{Feasibility Check: Recovery from the Exact Relative Distances}
\subsubsection{Recovery from exact distances}\label{sec:results:orientation-recovery:exact}

%\mdeff{Story: works perfectly despite no convexity guarantee and sampling.}
%\mdeff{I made it concise but precise. Let's do that for all!}

To verify that (i) the lack of a convexity guarantee for \eqnref{orientation-recovery} and (ii) the severe sampling of the sum are non-issues in practice, we attempt orientation recovery under exact distance estimation $d_p(\p_i, \p_j) = d_q(q_i, q_j)$.
%\mdeff{With $7000$ steps we actually don't under-sample, as $7000 \times 256 > P^2-P)/2 \approx \num{350e3}$.}
From $P_{test}=838$ projections taken from the asymmetric protein \texttt{5j0n}, we randomly sample batches of $256$ pairs (out of $(P^2-P)/2 \approx \num{350e3}$) at every step and minimize \eqnref{orientation-recovery} with the Adam optimizer~\cite{kingma2014adam} for $\num{7000}$ steps ($\sim 1.4$ hour) with a learning rate of $0.5$.
Orientations are perfectly recovered.
\figref{5j0n-orientation-recovery-loss} shows the convergence of~\eqnref{orientation-recovery} to zero.

\begin{figure}[ht!]
    \centering
    %\begin{subfigure}[b]{0.45\textwidth}
        \includegraphics[height=7cm]{figures/5j0n_perfect_angle_recovery}
        %\caption{Asymmetric protein (\texttt{5j0n}).}
    %\end{subfigure}
    %\hfill
    %\begin{subfigure}[b]{0.5\textwidth}
    %\centering
    %    \includegraphics[height=5.5cm]{figures/5a1a_perfect_angle_recovery}
    %    \caption{Symmetric protein (\texttt{5a1a}).}
    %\end{subfigure}
    \caption{
        Example of perfect orientation recovery (for \texttt{5j0n}).
        The loss~\eqnref{orientation-recovery} converges to zero when the distance estimation is perfect, i.e., $d_p(\p_i, \p_j) = d_q(q_i, q_j)$.
    }\label{fig:5j0n-orientation-recovery-loss}
\end{figure}

Empirically we find that the mean orientation recovery error \eqnref{orientation-recovery-error} $E = 0$.
The sphere coverage before and after the orientation alignment can be seen in the \figref{5j0n-aa-loss-perfect-distances}.
We can see that the orientation alignment was performed successfully.

\begin{figure}[ht!]
    \centering
    \begin{subfigure}[b]{0.45\textwidth}
        \includegraphics[height=6cm]{figures/coverage_alignment_before}
        \caption{Orientations before alignment.}
    \end{subfigure}
    \hfill
    \begin{subfigure}[b]{0.50\textwidth}
    \centering
        \includegraphics[height=6cm]{figures/coverage_alignment_after}
        \caption{Orientations after alignment.}
    \end{subfigure}
    \\
    \begin{subfigure}[b]{0.45\textwidth}
        \includegraphics[height=5.5cm]{figures/5j0n_perfect_angle_ralignment_before}
        \caption{Orientation recovery error without alignment.}
    \end{subfigure}
    \hfill
    \begin{subfigure}[b]{0.5\textwidth}
    \centering
        \includegraphics[height=5.5cm]{figures/5j0n_perfect_angle_ralignment_after}
        \caption{Orientation recovery error with alignment.}
    \end{subfigure}
    \caption{%
        Example of perfect alignment after a perfect orientation recovery under the true distance $d_p(\p_i, \p_j) = d_q(q_i, q_j)$ (\texttt{5j0n}).
        The first row shows the orientation coverage of $\mathbb{S}^2 \subset \SO(3)$ after the recovery.
        Green points are the ground truth orientations ${\{q_p\}}_{p=1}^P$ and red points are the recovered orientations ${\{\widehat{q_p}\}}_{p=1}^P$. The Figure (b) is exactly aligned which can be seen when zoomed. Due to plotting artifacts, both colors can still be seen. 
        % orientations projected on S²
        The second row are the histograms of ${\{ d_q (q_p, \T \widehat{q_p}) \}}_{p=1}^P$, with $\T = \mathbf{I}$ on the left (i.e., without alignment) and $\T$ as the optimum of \eqnref{orientation-recovery-error} on the right.
        Alignment is necessary to evaluate the performance of orientation recovery.
    }\label{fig:5j0n-aa-loss-perfect-distances}
%    \label{fig:angle-alignment-perfect}
\end{figure}


%\subsubsection{Robustness of Recovery to Additive Errors on the Relative Distances}
\subsubsection{Sensitivity to distance estimation error}\label{sec:results:orientation-recovery:sensitivity}

%\mdeff{Story: (i) orientation recovery error is strongly linked to distance estimation error, (ii) recovery loss is a good proxy of mean recovery error.}

We now go one step further and evaluate the behaviour of~\eqnref{orientation-recovery} when the true relative distances are corrupted by additive Gaussian noise.

The experimental conditions are the same as in the previous section, except that we add an error with increasing variance on the relative distances prior to the minimization. Precisely: $d_p = d_q + n$ with $n$ sampled from a Gaussian distribution with mean 0 and variances in $[0.0, 0.8]$.
The results are presented in \figref{perfect-with-noise-ar-aa} (red curve).
For all variances, the mean orientation recovery error $E$ is reported in \figref{perfect-with-noise-ar-aa} (blue curve).

\begin{figure}[ht!]
    \centering
    \begin{subfigure}[b]{0.48\textwidth}
        \includegraphics[height=5.5cm]{figures/5j0n_perfect_noisy_ar_aa}
        \caption{Asymmetric protein (\texttt{5j0n}).}
    \end{subfigure}
    \hfill
    \begin{subfigure}[b]{0.50\textwidth}
    \centering
        \includegraphics[height=5.5cm]{figures/5a1a_perfect_noisy_ar_aa}
        \caption{Symmetric protein (\texttt{5a1a}).}
    \end{subfigure}
    \caption{
        The mean orientation recovery error $E$ from \eqnref{orientation-recovery-error} is a monotonic function of the distance estimation error.
        Better distance estimation leads to better orientation recovery.
        Moreover, the recovery loss \eqnref{orientation-recovery} is a good proxy for the recovery error $E$, allowing us to assess recovery performance even without ground-truth orientations.
}
    \label{fig:perfect-with-noise-ar-aa}
\end{figure}

These results demonstrate that the performance of orientation recovery~\eqnref{orientation-recovery} depends on the quality of the estimated distances, which advocates for a proper and extensive training of the SiameseNN in the next stages of development.
Another interesting output of \figref{perfect-with-noise-ar-aa} is that it indicates that the error of the orientation recovery behaves as a monotonic function of its loss.
Hence, it suggests that the loss can be used as a good indicator of its performance, which has obvious practical implications for our future works on real data.

%%%%%%%%%%%%%%%%%%%%%%%%%%%%%%%%%%%%%%%%%%%%%%%%%%%%%%%%%%%%%%%%%%%%%%%%%%%%%%%%%%%%%%%
\subsection{Distance estimation}\label{sec:results:distance-estimation}
%\subsection{Estimating Relative Orientations from Projections}
%\subsection{Relative orientation estimation}

%\mdeff{Story: $d_p$ good estimator of $d_q$.
%SiameseNN better than l2, but still plateaus.
%Robust to projection noise.}

The challenge of distance estimation is to define the distance between two projections $d_p$ such that it is connected to the distance between two quaternions $d_q$. We start by using the Euclidean distance as the baseline. Then, we learn the distance metric using the SiameseNN architecture. With this network we aim to classify the new unseen pairs of projections without training the network again. Afterwards, we explore different network architectures as well as test the sensitivity of the network to perturbed projections.

\subsubsection{Euclidean distance}\label{sec:results:distance-estimation:euclidean}

%\mdeff{Story: simplest baseline estimator, $d_{pe}$ somewhat estimates $d_q$, quickly plateaus (even in the simplest noiseless and centered case).
%Note the difference between symmetric and asymmetric proteins.}

We evaluate $d_p(\p_i, \p_j) = \Vert \p_i - \p_j \Vert_2$ (i.e., the Euclidean distance) as a baseline distance estimator.
From $P = 5,000$ possible projection, we randomly select $5$ projections.
For each of these projections, we compute the Euclidean distance between aforementioned projection and all the others $d_p(\mathbf{p}_i,\mathbf{p}_j)=\lVert\mathbf{p}_i-\mathbf{p}_j\rVert_2$ and their corresponding orientation distance $d_q(q_i,q_j)$ through~\eqnref{distance:orientations}.
We then report the $(d_q,d_p)$ relationship for all pairs in \figref{euclidean-not-robust}, for both the asymmetric protein (left) and the symmetric one (right).

\begin{figure}[ht!]
    \centering
    \begin{subfigure}[t]{0.45\textwidth}
        \includegraphics[height=7.5cm]{figures/eucl_notrobust_5j0n}
        \caption{Asymmetric protein (\texttt{5j0n}).}
    \end{subfigure} \quad \quad
    \begin{subfigure}[t]{0.45\textwidth}
        \includegraphics[height=7.5cm]{figures/eucl_notrobust_5a1a}
        \caption{Symmetric protein (\texttt{5a1a}).}
    \end{subfigure}
    \caption{
        Plotting the Euclidean distance between two projections versus their actual relative orientation (measured by the geodesic distance) for \textbf{(left)} the asymmetric protein (\texttt{5j0n}) dataset, and \textbf{(right)} the symmetric protein (\texttt{5a1a}) dataset.
        The color corresponds to projection pairs that share the first projection \textit{i.e.} distance between one projection with all other projections.
        %\todo{Figure label: ${\{(0, p)\}}_{p=1}^P$?}
    }\label{fig:euclidean-not-robust}
\end{figure}

Two principal observations can be made from this experiment.
First, as suspected, $d_p$ fails to be a consistent predictor of $d_q$, even in the simple imaging conditions considered here (no noise, no translation, no PSF).
In particular, the larger the quaternion distance $d_q$, the poorer the predictive ability of $d_p$ (the plot plateaus).
The other interesting observation is that the trend of $(d_q,d_p)$ plot of the $\beta$-galactosidase protein (\texttt{5a1a}) appears to take symmetric shape of letter \texttt{M} which can be explained with the fact that this protein has intrinsic dihedral (D2) symmetry~\cite{noauthor_d2sym_nodate,noauthor_5a1asym_nodate}.
%\mdeff{How does it appear?}

\subsubsection{Learned distance}\label{sec:results:distance-estimation:learned}

%\mdeff{Story: learned distance $d_{ps}$ estimates $d_q$ with some variance but still underestimates larger distances.
%Again symmetric vs asymmetric.}

We present here a preliminary evaluation of the ability of SiameseNNs to learn a projection distance $\widehat{d}_p$ that correctly approximates the orientation distance $d_q$.

%SiameseNNs come with a variety of more or less powerful architectures.
%At the current stage of development, we work with a simple one.
%Our SiameseNN is composed of two convolutional neural networks (CNNs) with shared weights.
%Their output features vectors are compared through an Eulidean distance, \textit{i.e.}, $d_f(\mathbf{f}_i,\mathbf{f}_j)=\lVert \mathbf{f}_i-\mathbf{f_j}\rVert_2$ in \figref{schematic:distance-learning}.
%Besides the Euclidean distance, this distance metric $F$ can be defined as geodesic distance, or it could be parametrized as MLP, used for a general function approximation, which we will explore in some of the following experiments.
%\mdeff{Don't repeat what's written in \secref{method:distance-learning}. The general stuff goes there, the specific here.}

\begin{figure}
    \centering
    \begin{subfigure}[t]{0.45\textwidth}
        \includegraphics[height=6cm]{figures/de_5j0n}
        \caption{Asymmetric protein (\texttt{5j0n}).}
        \label{fig:losses-siamese-assym}
    \end{subfigure} \quad \quad
    \begin{subfigure}[t]{0.5\textwidth}
        \includegraphics[height=6cm]{figures/de_5a1a}
        \caption{Symmetric protein (\texttt{5a1a}).}
        \label{fig:losses-siamese-sym}
    \end{subfigure}
    \caption{
        Distance learning loss \eqnref{distance-learning} evaluated on the training and validation datasets during learning/training.
    }\label{fig:losses-siamese}
\end{figure}

\begin{figure}
    \centering
    \begin{subfigure}[b]{0.5\columnwidth}
        \includegraphics[height=6cm]{figures/dPdQ_5j0n}
        \caption{Asymmetric protein (\texttt{5j0n}) on test dataset.}
    \end{subfigure}
    %\hfill
    \begin{subfigure}[b]{0.45\columnwidth}
    \centering
        \includegraphics[height=6cm]{figures/dPdQ_5a1a}
        \caption{Symmetric protein (\texttt{5a1a}) on test dataset.}
    \end{subfigure}
    \caption{Relationship between orientations' distance $d_q$ and estimated distance $d_p$.}
    \label{fig:learned-distance-siamese}
\end{figure}

%\mdeff{What is $d_f$ in this particular experiment? Euclidean? Cosine=$d_q$?}
For each protein, we train the SiameseNN on its training dataset for 150 epochs ($\sim$2.6 hours) using an RMSProp optimizer~\cite{noauthor_tfkerasoptimizersrmsprop_nodate}, a learning rate of $10^{-3}$, and a batch size of 256 pairs. As a feature distance $d_f$ between the outputs of the two CNNs we use the Geodesic distance \eqnref{geodesic-distance}.
The pairs for the training are sampled from $1\%$ of maximum number of training pairs $P_{train}^2$ ($63,126$ pairs) and validation is performed on $1\%$ of the maximum number of validation pairs $P_{val}^2$ ($27,225$ pairs). We limit our training and validation dataset due to Google Colab training time limit of 12 hours.

%\mdeff{So those pairs are sampled from $63,126$ pairs from the training dataset, rather than the $P^2$ possible pairs? If true, we should motivate somewhere why we limit our training dataset.}
Depending on the available resources on the Google Colab, the training can last from 2.6 hours to 9.3 hours on one of its GPUs.
The evolution of the training and validation losses are presented in \figref{losses-siamese-assym} for the asymmetric protein (\texttt{5j0n}), and in \figref{losses-siamese-sym} for the symmetric one (\texttt{5a1a}).
The results demonstrate that the SiameseNN succeeds at learning a proxy distance for the asymmetric protein dataset, as convergence is reached in about 50 epochs ($\sim$ 50 minutes in the best resource availability setting).

%\todo{Mention the plateau phenomenon, similarly to Euclidean $d_p$.}
Similarly to Euclidean distance $d_p$, we notice that the larger distances $d_q$ are poorly predicted and the plot again has a slight plateau phenomenon for the distances higher than ~$2.5$ rad.

It is interesting that both validation losses are around $0.2$.
However, the current SiameseNN architecture slightly overfits at learning the distance for the dataset \texttt{5a1a}, which is very likely due to the symmetry of the $\beta$-galactosidase protein, even thought the quarter-sphere coverage was used.
Indeed, its synthetic dataset may still contain pairs of projections that share the same $d_p$, yet differ in their $d_q$.
This simply advocates for the restriction to non-overlapping areas on $\SO(3)$ when sampling the orientations used to generate the SiameseNN training dataset.
The latter would then only contain projection pairs with a linear $(d_q,d_p)$ relationship, which should ensure a successful training of the network.
%\mdeff{I don't get this explanation. Do you mean that \texttt{5a1a} might have other symmetries than D2?}
For the rest of the experiments, we use the asymmetric protein (\texttt{5j0n}) dataset.
Besides using the asymmetric protein, we will perform the full protein reconstruction pipeline on the symmetric protein (\texttt{5a1a}).

We then feed to the trained SiameseNN $1,000$ pairs of projections randomly selected from the \texttt{5j0n} testing dataset, and report the $(d_q,\widehat{d}_p)$ relationship of each pair in \figref{learned-distance-siamese}.
These results confirm that, the SiameseNN is able to predict the orientation distance $d_q$ using only the projections as inputs. The prediction performance is slightly better for the asymmetric protein compared to symmetric protein.
Moreover, it clearly outperforms the Euclidean distance at doing so.
These preliminary results are encouraging, as much has yet to be gained from improving upon the rather primitive SiameseNN architecture we currently use. The architecture of implemented SiameseNN can be seen in Appendix~\ref{sec:siamese-architecture}.

\subsubsection{Influence of network architecture and feature distance}

%\mdeff{Story: $d_f = d_q$ better than Euclidean and MLP $d_f$. Architecture of $G_w$ doesn't seem to matter much. Surprising, because we don't overfit $\rightarrow$ future research needed.}

To further improve our network, we experiment with different feature distance metrics $d_f(\mathbf{f}_i,\mathbf{f}_j)$.
In the previous experiments we used the Euclidean distance as a features distance metric, \textit{i.e.} $d_f(\mathbf{f}_i,\mathbf{f}_j)=\lVert \mathbf{f}_i-\mathbf{f}_j\rVert_2$.
%\mdeff{Consistency: In \secref{method:distance-learning}, I used $d_f$  (for consistency with $d_p$ and $d_q$) and $\mathbf{f}_i$ (for consistency with $\p_i$ and $q_i$).}

In this experiment, we test the performance with a geodesic distance, \textit{i.e.}
\begin{equation*}
    d_f(\mathbf{f}_i,\mathbf{f}_j) = 2 \arccos \left( \frac{\mathbf{f}_i \cdot \mathbf{f}_j}{\lVert \mathbf{f}_i \rVert \lVert \mathbf{f}_j \rVert} \right) = 2 \arccos \left( \frac{\sum_{k=1}^n f_{i,k} f_{j,k}}{\sqrt{\sum_{k=1}^n f_{i,k}^2}\sqrt{\sum_{k=1}^n f_{j,k}^2 }} \right).
    \label{eqn:geodesic-distance}
\end{equation*}

Additionally, we parametrize $d_f(\mathbf{f}_i,\mathbf{f}_j)$ as MLP. It consists of six hidden linear layers with \texttt{1024}, \texttt{512}, \texttt{512}, \texttt{256}, \texttt{256}, and \texttt{1} unit respectively. All of them use SeLU as an activation function.

The experimental conditions are the same as in the previous section with the Euclidean distance metric as feature distance metric.
\figref{geo-eucl-mlp} shows the training and validation losses of the geodesic distance are better than Euclidean and MLP feature distance metric. We can also observe the performance on the $(d_q, d_p)$ plots and notice that the projection pairs deviate the least using the geodesic distance.
%\todo{Why? Add and rephrase hypotheses. Cosine distance has the same geometry as $d_q$. Euclidean has not (no periodicity nor curvature). MLP doesn't guarantee it's a distance function (not symmetric, zero between itself, triangle inequality) and might need much more data.}
The reason for a geodesic distance to perform the best is that it has the same geometry as quaternion distance $d_q$ in $\mathbf{SO}(3)$ space. Whereas, the Euclidean distance is not desirable on $\mathbf{SO}(3)$ since it does not respect the manifold's non-linearity (it has no periodicity nor curvature) and can lead to unpredictable behaviours. The $\mathbf{SO}(3)$ is non-linear and it can be explained with the fact that Euclidean distance of two quaternions can be small, despite the rotation being large~\cite{huynh_metrics_2009,DBLP:journals/corr/abs-1805-01026}.
Moreover, MLP does not guarantee that it is a distance function. For MLP to be a distance function, it should have the following properties: (1) support symmetry; (2) output zero with two identical inputs; (3) support triangle inequality. In addition, we might need much more data for the MLP training.
Hence, it is desirable to have a distance function that respects the structure of $\mathbf{SO}(3)$ space. 

\begin{figure}
    \centering
    \includegraphics[height=7cm]{figures/geo_eucl_mlp_distance_metric.pdf}
    \caption{
        Training and validation epoch losses w.r.t. feature vectors' distance metric and their corresponding distance ratio $(d_q, d_p)$ plots.
    }\label{fig:geo-eucl-mlp}
\end{figure}

\subsubsection{Sensitivity to perturbed projections}\label{sec:results:distance-estimation:sensitivity}

%\mdeff{Story: learned distance is minimally sensible to perturbations (additive noise, translation, PSF) because we can train it to ignore irrelevant information.
%Thanks again to good model of cryo-EM imaging.}
%\mdeff{Better word? (perturbations, quality, non-ideal)}

In this experiment we want to explore how does the perturbation of the projections affect the distance estimation (additive noise, translation, PSF).

The experimental conditions for the experiment are the same as before, except that we add noise with increasing variance on the projection prior to training.
The results are presented in \figref{distance-estimation-vary-projection-noise}.

We observe that the training and validation losses are increasing and network starts to overfit w.r.t.\ the amount of noise in the projections.
With the noiseless projections (projection noise variance 0), the mean orientation recovery error $E = 0.1594$ rad.
Whereas, the noisy projections with noise levels 15 are the closest to the realistic protein projections and the error $E=0.4189$ rad.
The SiameseNN is able to learn the noise as the training loss and corresponding mean orientation recovery error $E$ stays low.
%\mdeff{The SiameseNN seems to be able to learn the noise, as the training loss stays low. More data should help!}

Besides testing the performance of the pipeline with the noisy projections, we explore the performance with different projection translation levels.
To translate the projection, we use a triangular distribution from $-t$ to $t$ px translation with the peak in the center of the projection.
The performance of different translation magnitudes can be seen in \figref{distance-estimation-vary-projection-translation}.
We observe that the training of the network is invariant to translations in the projections, which was expected.
%\mdeff{Perfect!}

We can see that the learned distance is minimally sensible to perturbations because we can train the network to ignore irrelevant information.


\begin{figure}[ht!]
    \centering
    \begin{subfigure}[b]{0.47\textwidth}
        \includegraphics[height=5.5cm,valign=t]{figures/de_noises_nums}
        \caption{%
            Variation of train and validation epoch losses w.r.t. noise levels in the projections of the asymmetric protein (\texttt{5j0n}). The mean orientation recovery error $E$ is in the red box, and the orientation recovery loss $OR$ is in the blue box.
        }\label{fig:distance-estimation-vary-projection-noise}
    \end{subfigure}
    \hfill
    \begin{subfigure}[b]{0.47\textwidth}
        \includegraphics[height=5.5cm,valign=t]{figures/de_translation_nums}
        \caption{
        Variation of train and validation epoch losses w.r.t.\ projection translation of the asymmetric protein (\texttt{5j0n}). The mean orientation recovery error $E$ is in the red box, and the orientation recovery loss $OR$ is in the blue box.
        %\mdeff{We should try to have 12 and 13 side-by-side (to gain some space and facilitate comparison on the y-axis) by making them more square.}
    }\label{fig:distance-estimation-vary-projection-translation}
    \end{subfigure}
\end{figure}

\subsection{Orientation recovery from estimated distances}

%\mdeff{Story: pipeline works but better distance estimation is needed for SOTA reconstruction.
%Method is however promising because learned distance is robust to perturbations and recovery works if distance works.}
%\todo{Justify threshold because of plateau (figref).
%Show recovered orientations w.r.t.\ ground truth after alignment.}
%\todo{Reconstruct the protein to show the full pipeline: from a set of projections to a reconstructed protein.
%Emphasize that it's a naive reconstruction algorithm.}

The orientation recovery from estimated distances represents a full pipeline needed to reconstruct the protein from a given set of projections.
We run the pipeline for both, asymmetric (\texttt{5j0n}) and symmetric (\texttt{5a1a}) protein.
In addition, we run the pipeline for the simulated realistic noise in the asymmetric protein.
The experimental setting for distance estimation is the similar to the one used to generate the \figref{learned-distance-siamese}: 150 epochs, 1e-3 learning rate, batch size 256 with random sampling of the projections, but for feature distance metric we use the geodesic distance since it showed the best performance in \figref{geo-eucl-mlp}.

Then, we run the orientation recovery on the estimated distances of the asymmetric protein and the performance results can be observed in \figref{5j0n-orientation-recovery-loss-est} with the same experimental setting as in \figref{5j0n-orientation-recovery-loss}.
With the noiseless projections, the objective function successfully converges to the $0.0510$ and with the noisy projections, the objective function converges to the $0.0683$.

\begin{figure}[ht!]
    \centering
    \begin{subfigure}[b]{0.45\textwidth}
        \includegraphics[height=5.5cm]{figures/5j0n_noise0_angle_recovery}
        \caption{Recovery loss, noiseless projections $\mathbf{Px}$.}
    \end{subfigure}
    \hfill
    \begin{subfigure}[b]{0.5\textwidth}
    \centering
        \includegraphics[height=5.5cm]{figures/5j0n_noise16_angle_recovery}
        \caption{Recovery loss, noisy projections $\mathbf{Px+n}, \; \mathbf{n} \sim \mathcal{N}(0, 16\mathbf{I})$.}
    \end{subfigure}
    \\
    \begin{subfigure}[b]{0.45\textwidth}
    \centering
        %\includegraphics[height=5.7cm]{figures/5j0n_noise0_angle_alignment_before}
        \includegraphics[height=5.7cm]{figures/5j0n_noise0_angle_alignment_after}
        \caption{Recovery error, noiseless projections $\mathbf{Px}$.}
        \label{fig:angle-alignment-5j0n-noise0}
    \end{subfigure}
    \hfill
    \begin{subfigure}[b]{0.5\textwidth}
    \centering
        %\includegraphics[height=5.7cm]{figures/5j0n_noise16_angle_alignment_before}
        \includegraphics[height=5.7cm]{figures/5j0n_noise16_angle_alignment_after}
        \caption{Recovery error, noisy projections $\mathbf{Px+n}, \; \mathbf{n} \sim \mathcal{N}(0, 16\mathbf{I})$.}
        \label{fig:angle-alignment-5j0n-noise16}
    \end{subfigure}
    \caption{%
        Performance of orientation recovery of the asymmetric protein (\texttt{5j0n}) with (right) and without (left) noise.
        The first row shows the orientation recovery loss.
        The second row shows the orientation recovery error ($E$ from \eqnref{orientation-recovery-error}).
    }\label{fig:5j0n-orientation-recovery-loss-est}
\end{figure}

The mean orientation recovery error for asymmetric protein \texttt{5j0n} without noise in the projection can be seen in \figref{angle-alignment-5j0n-noise0}.
The smallest error achieved is $0.1594$ rad.
The mean orientation recovery error for asymmetric protein \texttt{5j0n} with noisy projections (white noise with variance 16) can be seen in \figref{angle-alignment-5j0n-noise16}. 
The smallest error achieved is $0.4184$ rad.

As a last step of the pipeline, we perform protein reconstruction using the projections and their corresponding estimated orientations.
For that, we again use the ASTRA toolbox.
Using this toolbox, we generate orientation vectors based on angles which we later feed into projection 3D geometry in ASTRA.
Due to GPU memory limit, we are able to reconstruct the protein using maximum of $3,000$ projections.
It holds for our case, since we perform orientation recovery on the test set which has in total $1,650$ projections (less than the limit of $3,000$).

The reconstruction results for the asymmetric protein (\texttt{5j0n}) with noiseless projections can be seen in \figref{5j0n-reconstruction-noise0}. On the left side we have a reconstruction using the ground-truth orientations, and on the right side we have the reconstruction result using the estimated aligned orientations.

\begin{figure}[ht!]
    \centering
    \begin{subfigure}[b]{0.49\linewidth}
        \centering
        \includegraphics[width=0.90\linewidth]{figures/5j0n_reconstruction_GT}
        \caption{Noiseless projections $\mathbf{Px}$, true orientations ${\big\{q_p\big\}}_{p=1}^P$.}
    \end{subfigure}
    \hfill
    \begin{subfigure}[b]{0.49\linewidth}
        \centering
        \includegraphics[width=0.90\linewidth]{figures/5j0n_reconstruction_GT_noise16}
        \caption{Noisy projections $\mathbf{Px + n}$, true orientations ${\big\{q_p\big\}}_{p=1}^P$.}
    \end{subfigure}
    \\
    \begin{subfigure}[b]{0.49\linewidth}
        \centering
        \includegraphics[width=0.90\linewidth]{figures/5j0n_reconstruction_noise0}
        \caption{Noiseless projections $\mathbf{Px}$, recovered orientations ${\big\{\widehat{q_p}\big\}}_{p=1}^P$.}
    \end{subfigure}
    \hfill
    \begin{subfigure}[b]{0.49\linewidth}
        \centering
        \includegraphics[width=0.90\linewidth]{figures/5j0n_reconstruction_noise16}
        \caption{Noisy projections $\mathbf{Px + n}$, recovered orientations ${\big\{\widehat{q_p}\big\}}_{p=1}^P$.}
    \end{subfigure}
    \caption{
        Performance of orientation recovery of the asymmetric protein (\texttt{5j0n}) with (right) and without (left) noise.
        Reconstruction of the asymmetric protein (\texttt{5j0n}) from noiseless (left) and noisy (right) projections ${\big\{\p_i\big\}}_{i=1}^P$ and true (top) and recovered (bottom) orientations.
    }\label{fig:5j0n-reconstruction-noise0}
    \label{fig:5j0n-reconstruction-noise16}
\end{figure}

The reconstruction results for the asymmetric protein (\texttt{5j0n}) with noisy projections (with noise variance 16) can be seen in \figref{5j0n-reconstruction-noise16}.
Similarly, on the left side we have a reconstruction using the ground-truth orientations, and on the right side we have the reconstruction result using the estimated aligned orientations.

Similarly, we run the whole reconstruction pipeline on the symmetric protein (\texttt{5a1a}).
The experimental conditions for the distance estimation are the same as for the asymmetric protein, except that we use the quarter-sphere projections coverage (whereas, in the asymmetric protein we use half-sphere coverage).
The orientation recovery loss can be seen in \figref{5a1a-orientation-recovery-loss}. It successfully converges to $0.0381$.
The mean orientation recovery error for symmetric protein \texttt{5a1a} can be seen in \figref{angle-alignment-5a1a-noise0}. The smallest error achieved is $0.1871$ rad.

\begin{figure}[ht!]
    \centering
    \begin{subfigure}[b]{0.45\textwidth}
        \centering
        \includegraphics[height=5.5cm]{figures/5a1a_noise0_angle_recovery}
        \caption{Orientation recovery loss \eqnref{orientation-recovery}.}
    \end{subfigure}
    \hfill
    \begin{subfigure}[b]{0.45\textwidth}
        \centering
        \includegraphics[height=5.5cm]{figures/5a1a_ground_truth}
        \caption{Reconstruction from true orientations ${\big\{q_p\big\}}_{p=1}^P$.}
    \end{subfigure}
    \\
    \begin{subfigure}[b]{0.45\textwidth}
        \centering
        %\includegraphics[height=5.7cm]{figures/5a1a_noise0_angle_alignment_before}
        \includegraphics[height=5.5cm]{figures/5a1a_noise0_angle_alignment_after}
        \caption{Orientation recovery error \eqnref{orientation-recovery-error}.}
    \end{subfigure}
    \hfill
    \begin{subfigure}[b]{0.45\textwidth}
        \centering
        \includegraphics[height=5.5cm]{figures/5a1a_aligned}
        \caption{Reconstruction from recovered orientations ${\big\{\widehat{q_p}\big\}}_{p=1}^P$.}
    \end{subfigure}
    \caption{
        Orientation recovery and reconstruction of the symmetric protein (\texttt{5a1a}) from noiseless projections.
    }\label{fig:5a1a-orientation-recovery-loss}
    \label{fig:angle-alignment-5a1a-noise0}
    \label{fig:5a1a-reconstruction-noise0}
\end{figure}

Lastly, we perform the protein reconstruction with the same ASTRA toolbox setting as for the asymmetric protein. The results of the reconstruction can be observed in the \figref{5a1a-reconstruction-noise0}.
We were able to successfully reconstruct the symmetric protein even though the distance estimation was noisier than the one performed on the asymmetric protein.

We observe that the pipeline works, but for the state-of-the-art reconstruction we need a better distance estimation.
However, the method developed is promising since the learned distance is robust to perturbations.
We observe that the orientation recovery and distance estimation are interconnected, \textit{i.e.} if one works the other one will work.

\section{Discussion}\label{sec:discussion}

% Summary.
In this work, we explored the use of distance learning between pairs of 2D cryo-EM projections from a 3D protein structure to infer the unknown orientation at which each projection was imaged from.
Our two-step method relies on the training of a SiameseNN to estimate pairwise distances between unseen projections, followed by the recovery of the orientations from these distances through an appropriate minimization scheme.

The benefit of this approach, at least in theory, is that it would permit to estimate the unknown orientations in single-particle cryo-EM directly from the acquired dataset, \ie, without the need for intermediate reconstruction procedure or initial volume estimate; this has obvious attractive implications in the field.

At the current stage of development, the method has been evaluated on synthetic datasets for two different proteins.
The results provide key insights on the viability of the proposed scheme.
First, they demonstrate that a SiameseNN can learn a distance function between projections that estimates the difference in their orientation (\secref{results:distance-estimation:learned}) and that is invariant to off-centering shifts and robust to increasing levels of noise (\secref{results:distance-estimation:sensitivity})---an important condition in cryo-EM\@.
%\mdeff{The two phrases in the following sentence sound redundant.}
Second, they guarantee that an accurate distance estimation leads to a correct recovery of the orientations. % , while providing indications that the quality of this recovery depends on the precision of the estimated distances (\secref{results:orientation-recovery:sensitivity}, \secref{results:distance-estimation:sensitivity}).
Finally, our method was able to recover orientations with an error of \banjac{$0.12$ to $0.19$ radians ($6$ to $11\degree$) from noiseless and $0.13$ to $0.25$ radians ($7$ to $14$\degree) from noisy projections}---leading to an initial \todo{volume/reconstruction} with a resolution of \banjac{$8.0$ to $9.6$\AA\ for the symmetric protein, and $12.2$ to $15.2$\AA\ for the asymmetric protein when the FSC is $0.5$} \mdeff{or ``with a resolution 4x worse that the ground-truth?''} (\secref{results:orientation-recovery:reconstruction}).
In summary, the more accurate the estimated distances, the more precise the recovery of the orientations, and, ultimately, the better the reconstructed volume.

% Future work.
While the method is not yet at the stage where it can be deployed in practice, we believe that a series of developments could help it become a more relevant contributor for single-particle cryo-EM reconstruction.%
\footnote{Note that the present project will not be further continued by its authors due to other professional occupations. Hence, we strongly encourage anyone interested to build on these ideas and, hopefully, make it a practical tool.}
As previously discussed, the results underline the importance of learning an accurate distance estimator. % $\widehat{d_p}$.
In this regard, the performance of the SiameseNN could be improved in several ways.
% Method gains: mostly distance learning maybe recovery (not alignment).
%A set of additional technical developments could also further improve performance.
First, the architecture of the SiameseNN's twin CNNs should be expanded and tuned.
% While it gave good performance for a reasonable runtime, it could be optimized.
% , as well as the distance metric between the two CNN outputs.
% For instance, one could parametrize the function $d_f$---which compares the similarity of the features outputs (see \figref{schematic:distance-learning})---as a feed-forward neural network instead of the current Euclidean distance, and learn its weights as well.
Second, the training of the SiameseNN could be improved, perhaps by providing more supervision by separately predicting the differences in direction $(\theta_2,\theta_1)$ and in-plane angle $\theta_3$.
\todo{It can definitely learn mirroring / full coverage, albeit not as well as half (because it doesn't exploit that symmetry). Motivation to incorporate this physical knowledge into the NN architecture. Motivation (on top of half in-plane being much easier for 5a1a) to predict the direction and in-plane angle separately.}

% \mdeff{Let's ignore improvements to recovery and concentrate the story on better distance learning -> better recovery -> better reconstruction.}
%Among others, one shall explore whether reducing the influence of larger distances in orientation recovery could bring further gain in accuracy.
% \mdeff{The following are issues with alignment---not directly related to our method---that we mentioned elsewhere.}
% Angle alignment didn't always work, even when $L_\text{OR}$ was low (examples?): We might miss a transformation in \eqnref{orientation-recovery-error}.
% Why did we need to align with \eqnref{orientation-recovery-error} before reconstructing with ASTRA\@?

% Data gains.
Importantly, the SiameseNN would be better trained on a more exhaustive and diverse cryo-EM dataset.
Indeed, the success of the SiameseNN as a faithful estimator of relative orientations eventually relies on our capacity to generate a synthetic training dataset whose data distribution is diverse enough to cover that of unseen projection datasets.
Such realistic cryo-EM projections could be generated by relying on a more expressive formulation of the cryo-EM physics and taking advantage of the thousands of atomic models available in the PDB\@.
% \mdeff{PDB database sounds redundant as PDB stands for protein database.}
In particular, a necessary extension will be to include the effects of the PSF when generating training data and evaluate its impact on the SiameseNN\@. % like we did for shifts and noise

% Towards practical use: unseen proteins (also data) and real measurements.
% A final phase of tests before deploying the method on real cryo-EM measurements will be to extensively test the method on ``unseen proteins'', \ie, proteins whose simulated projections have never been seen by the SiameseNN\@.
% Early experiments indicate the feasibility of this enterprise (see \apxref{unseen-proteins}).
% In this regard, an interesting aspect of our method is that the twin CNNs within the SiameseNN intrinsically predict the \textit{relationship} between projections, allowing the SiameseNN as a whole to abstract the particular volume.
% %Consequently, a well-trained distance estimator could be relatively robust to the ``mismatch'' of volumes within the training set.
% %In the same line of thought,
% Learning benefits from the profound structural similarity shared by proteins---after all, they are all derived from the same $21$ building blocks.
% \todo{remove Appendix \apxref{unseen-proteins}}
% Learning exploits statistical effects, given here by biological building block.

% \mdeff{I propose to omit the following sentence (which is a truism and is stated in the previous paragraph) to finish on a more forceful note.}
% Eventually, the performance of the method on real cryo-EM measurements will provide the real measure of its potential, the imaging conditions being notoriously challenging in single-particle cryo-EM\@.
% Further down the line and still in the real of the hypothetical, new approaches for the training of the SiameseNN able to handle the handling of proteins with multiple conformational states could be explored.


\section*{Acknowledgements}

Grants to acknowledge? \lau{I need to check with MU on my side.}

The authors are thankful to Dr. Matthieu Simeoni and Dr. Julien Fageot for insightful mathematical discussions throughout the project. 





\bibliographystyle{IEEEtran}
\bibliography{refs}

\clearpage

\section{Different performance metrics}\label{apx:metrics-review}

\todo{Keep? Review and copy-edit necessary.}

There are many different ways of evaluating the pipeline performance found in the field of pose estimation. Some of the evaluations include the following:
\begin{itemize}
\item Intersection over Union (IoU) of the object 3D cloud with a custom threshold classifying it as a good estimate or not (e.g. in the paper~\cite{10.1007/s11263-014-0733-5} the threshold score above 0.5 is considered good estimation).
\item Translation and rotation error between estimated 3D model and true 3D model with fixed thresholds (e.g. in the paper~\cite{shotton2013scene} they require the translation error to be below 5 cm and rotation error to be below 5\degree)
\item The average distance of all the points of the model from their transformed version, and if the error is less than the constant multiple of diameter of the 3D model, it is considered correctly evaluated (e.g. evaluation error is used in papers \cite{10.1007/978-3-642-37331-2_42, xiang2018posecnn})
\item Reprojection error that projects the estimated points onto the image and computes the pairwise distances in the image space, instead of computing distances in the 3D model space (e.g. used in paper~\cite{xiang2018posecnn})
\item The recovery error measured as Frobenius norm from estimated 3D model and true model, where 3D model is composed of 3D locations of important landmarks (e.g. elbow for human pose estimation)~\cite{wangni2018monocular}
\item Average Orientation Similarity (AOS) is the difference between the true and estimated model with a cosine similarity term~\cite{RedondoCabrera2016PoseEE}
\item Mean Angle Error (MAE) and Median Angle Error (MedError) evaluated and compared with other pose estimation error metrics in the paper~\cite{RedondoCabrera2016PoseEE}.
\end{itemize}

\section{Orientation recovery from exact distances}\label{apx:results:orientation-recovery:exact}

%\mdeff{Story: works perfectly despite no convexity guarantee and sampling.}

\lau{Smooth.}
To verify that the lack of a convexity guarantee for \eqnref{orientation-recovery} and the sampling of the sum are non-issues in practice, we attempted orientation recovery under exact distance estimation $d_p(\p_i, \p_j) = d_q(q_i, q_j)$.
Orientations were perfectly recovered.
\figref{5j0n-orientation-recovery-loss} shows the convergence of the loss $L_\text{OR}$ to zero.
\figref{5j0n-aa-loss-perfect-distances} shows how~\eqnref{orientation-recovery-error} could then perfectly align the recovered and true orientations---leading to $E_\text{OR} = 0$.
% Demonstrating that alignment is necessary to evaluate the performance of orientation recovery.

\begin{figure}[ht!]
    \begin{minipage}[t]{0.27\linewidth}
        \centering
        \includegraphics[height=3cm]{figures/5j0n_perfect_angle_recovery}
        \caption{%
            Example of perfect orientation recovery (for \texttt{5j0n}).
            The loss $L_\text{OR}$ \eqnref{orientation-recovery} converges to zero when the distance estimation is perfect, i.e., $d_p(\p_i, \p_j) = d_q(q_i, q_j)$.
            \todo{Add $L_\text{OR}$ to the y-axis label and blue box.}
        }\label{fig:5j0n-orientation-recovery-loss}
    \end{minipage}
    \hfill
    \begin{minipage}[t]{0.70\linewidth}
%        \begin{subfigure}[b]{0.19\linewidth}
%            \centering
%            \includegraphics[height=3cm]{figures/5j0n_perfect_angle_ralignment_after}
%            \caption{Orientation recovery error with alignment.}
%        \end{subfigure}
%        \hfill
        \begin{subfigure}[t]{4.3cm}
            \centering
            \includegraphics[height=3cm]{figures/5j0n_perfect_angle_ralignment_before}
            \caption{Error histogram $\{ d_q (q_i, \widehat{q_i}) \}$, i.e., before alignment.}
        \end{subfigure}
        \hfill
        \begin{subfigure}[t]{3.4cm}
            \centering
            \includegraphics[height=3cm]{figures/coverage_alignment_before.png}
            \caption{Orientations before alignment.}
        \end{subfigure}
        \hfill
        \begin{subfigure}[t]{3.4cm}
            \centering
            \includegraphics[height=3cm]{figures/coverage_alignment_after.png}
            \caption{Orientations after alignment.}
        \end{subfigure}
        \caption{%
            Example of perfect alignment~\eqnref{orientation-recovery-error} after a perfect orientation recovery~\eqnref{orientation-recovery}.
            (a)~shows the error before alignment.
            % and $\T$ as the optimum of \eqnref{orientation-recovery-error} on the right.
            (b-c)~show the orientations projected on $\mathbb{S}^2 \subset \SO(3)$ before (b) and after (c) alignment.
            Green points are the true orientations $\{q_i\}$ and red points are the recovered orientations $\{\widehat{q_i}\}$.
            % The Figure (b) is exactly aligned which can be seen when zoomed. Due to plotting artifacts,
            While both colors are seen in (c), they are exactly superimposed.
        }\label{fig:5j0n-aa-loss-perfect-distances}
    %    \label{fig:angle-alignment-perfect}
    \end{minipage}
\end{figure}

\section{Euclidean distance between projections}\label{apx:results:distance-estimation}
% Orientation distance as, Estimating distances with

%\mdeff{Story: simplest baseline estimator, $d_{pe}$ somewhat estimates $d_q$, quickly plateaus (even in the simplest noiseless and centered case).
%Note the difference between symmetric and asymmetric proteins.}

\begin{itemize}
    \item \mdeff{Jelena: from which coverage were the projections sampled from here?}
    \item \todo{Copy-edit.}
\end{itemize}

We evaluate $d_p(\p_i, \p_j) = \Vert \p_i - \p_j \Vert_2$ (i.e., the Euclidean distance) as a baseline distance estimator.
From $P = 5,000$ possible projection, we randomly select $5$ projections.
For each of these projections, we compute the Euclidean distance between aforementioned projection and all the others $d_p(\mathbf{p}_i,\mathbf{p}_j)=\lVert\mathbf{p}_i-\mathbf{p}_j\rVert_2$ and their corresponding orientation distance $d_q(q_i,q_j)$ through~\eqnref{distance:orientations}.
We then report the $(d_q,d_p)$ relationship for all pairs in \figref{euclidean-not-robust}, for both the \texttt{5j0n} (left) and \texttt{5a1a} (right).

Two principal observations can be made from this experiment.
First, as suspected, $d_p$ fails to be a consistent predictor of $d_q$, even in the simple imaging conditions considered here (no noise, no shift, no PSF).
In particular, the larger the quaternion distance $d_q$, the poorer the predictive ability of $d_p$ (the plot plateaus).
The other interesting observation is that the trend of $(d_q,d_p)$ plot of the \texttt{5a1a} appears to take symmetric shape of letter \texttt{M} which can be explained with the fact that this protein has intrinsic dihedral (D2) symmetry~\cite{noauthor_d2sym_nodate,noauthor_5a1asym_nodate}.
\mdeff{Check these refs. Other ones are used in main text?}

\begin{figure}[ht!]
    \begin{minipage}[t]{0.55\linewidth}
        \begin{subfigure}[t]{0.48\textwidth}
            \centering
            \includegraphics[height=4cm]{figures/eucl_notrobust_5j0n}
            \caption{\texttt{5j0n}}
        \end{subfigure}
        \hfill
        \begin{subfigure}[t]{0.48\textwidth}
            \centering
            \includegraphics[height=4cm]{figures/eucl_notrobust_5a1a}
            \caption{\texttt{5a1a}}\label{fig:euclidean-not-robust:5a1a}
        \end{subfigure}
        \caption{%
            The Euclidean distance between two projections $d_p(\p_i, \p_j) = \Vert \p_i - \p_j \Vert_2$ versus their actual relative orientation $d_q(q_i, q_j)$.
            Each color represent the distances between one fixed projection and the other $P-1$ projections.
    %        The color corresponds to projection pairs that share one projection, i.e., distance between one projection with all other projections.
            While there is some correlation, especially at small distances, the Euclidean distance is a poor estimator.
            Because \texttt{5a1a} has D2 symmetries, two projections might be identical while not having been acquired from the same orientation.
        }\label{fig:euclidean-not-robust}
    \end{minipage}
    \hfill
    \begin{minipage}[t]{0.4\linewidth}
        \centering
        \includegraphics[height=4cm]{figures/geo_eucl_mlp_distance_metric.pdf}
        \caption{%
            Performance of distance learning w.r.t.\ the choice of feature distance function $d_f$.
            The box plots show the distance learning loss $L_\text{DE}$ \eqnref{distance-learning} on the training (blue) and validation (red) sets.
            The inserted plots show the relationship between $d_q(q_i, q_j)$ and $d_p(\p_i, \p_j) = d_f(\mathcal{G}_w(\p_i), \mathcal{G}_w(\p_j))$.
            \todo{Consistency: train -> training set, validation -> validation set, and Geodesic -> Cosine. Add $L_\text{DE}$ to the y-axis label. Larger text.}
            \todo{Remove the MLP results. It didn't work at all. Make the difference between Euclidean and Cosine easier to see.}
        }\label{fig:geo-eucl-mlp}
    \end{minipage}
\end{figure}

\section{Choice of feature distance}\label{apx:feature-distance}

%\mdeff{Story: $d_f = d_q$ better than Euclidean and MLP $d_f$. }

There are multiple options for a distance function $d_f$ between two features $\mathbf{f}_i = \mathcal{G}_w(\p_i) \in \R^{n_f}$. We compared the performance of three different distances:

\begin{itemize}
    \item the Euclidean distance, i.e., $d_f(\mathbf{f}_i, \mathbf{f}_j) = \| \mathbf{f}_i - \mathbf{f}_j \|_2$
    \item the cosine distance, i.e.,
$ d_f(\mathbf{f}_i,\mathbf{f}_j) = 2 \arccos \left( \frac{\langle \mathbf{f}_i, \mathbf{f}_j \rangle}{\lVert \mathbf{f}_i \rVert \lVert \mathbf{f}_j \rVert} \right)$
\todo{No absolute value. So not as in \eqnref{distance:orientations}.}
    \item the parametrization of $d_f$ as a multi-layer perceptron (MLP) whose parameters are learned.
\end{itemize}

The MLP we tested was made of six layers with $1024, 512, 512, 256, 256, 1$ units, all of which use the SeLU activation function. The features $\mathbf{f}_i$ and $\mathbf{f}_j$ were stacked as an array of size $2n_f$ before being fed to the MLP\@. Note that, while a MLP can approximate any function, there is no guarantee that the learned function will satisfy the axioms of a proper distance function (i.e., the identity of indiscernibles, symmetry, and triangle inequality). \lau{Say that we were not able to train.}

\figref{geo-eucl-mlp} compares the use of the first two distances. The Euclidean distance yields better results, but the cosine distance is ultimately the best performer of all: $L_\text{DE}$ is the lowest, which makes $d_p$ a better estimator of $d_q$.
% the projection pairs deviate the least from the identity
This superiority of the cosine distance is likely due to its capacity to model the elliptic geometry of $\SO(3)$, a feat the Euclidean distance does not achieve, the Euclidean space being neither periodic nor curved.
%The $\SO(3)$ is non-linear and it can be explained with the fact that Euclidean distance of two quaternions can be small, despite the rotation being large~\cite{huynh_metrics_2009,DBLP:journals/corr/abs-1805-01026}.

\clearpage
\section{SiameseNN architecture}\label{apx:siamese-architecture}

\begin{figure}[h!]
    \centering
    \includegraphics[height=19cm]{figures/model_plot.png}
    \caption{%
        Distance estimation network architecture.
        We have two input images of dimensions $116 \times 116$.
        Each one goes to its CNN (part where we share the weights).
        The output is a scalar value representing the distance between these two images.
    }\label{fig:de-architecture}
\end{figure}


\end{document}
