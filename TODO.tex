
%%%%%%%%%%%%%%%%%%%%%%%%%%%%%%%%%%%%%%%%%%%%%%%%%%%%%%%%%%%%%%%%%%%%

\todo{\paragraph{Overall TODOs.}\begin{itemize}
    \item Use $\mathbf{p}$ to denote projections everywhere (also double-check figures, plot axes, legends, etc.).
    \item Index all objects using subscripts. Report all results in radians. Use sentence case. Use American english.
    \item Laurène is in charge of the Abstract, Introduction and Section 2.
    \item Jelena is in charge of Section 3, including the task of adding and describing the latest results.
\end{itemize}}

%%%%%%%%%%%%%%%%%%%%%%%%%%%%%%%%%%%%%%%%%%%%%%%%%%%%%%%%%%%%%%%%%%%%

\todo{\paragraph{STILL TO DISCUSS.}\begin{itemize}
    \item Bold face for vectors and matrices? \mdeff{That's quite signal processing. Is the cryo-EM community using this convention?} \lau{They do as well.}

    \item  Orientations: Use $q$ or $\boldsymbol\theta$ for the orientation, and $\mathbf{R}$ for rotation. To be settle after writing \secref{method:orientation-representation}.

    \item distance functions: orientations SO(3) geodesic $d_o$, projections Euclidean $d_{pe}$, projections learned $d_{pl}$. \lau{Let's settle on that.}

    \item Consistency: convergence curves must all have the same x-axis (time, step, or epoch). \mdeff{I think epoch would be good.}

    \item SO(3) half coverage (it is $S^2$ half coverage)

\end{itemize}}

%%%%%%%%%%%%%%%%%%%%%%%%%%%%%%%%%%%%%%%%%%%%%%%%%%%%%%%%%%%%%%%%%%%%

\todo{\paragraph{TO KEEP IN MIND.}\begin{itemize}
    \item phrases to keep all along: distance estimation (verb: estimate distances), distance learning (verb: learn distance), orientation recovery (verb: recover orientations)
    \item Typography: \mdeff{footnotes after periods, em dashes, Oxford comma, commas after i.e.\ and e.g., (i) and (ii). Again no strong feeling, most important is consistency.}
    \item Give meaningful and consistent names to figures. To avoid confusion, label = figure name.
\end{itemize}}

%%%%%%%%%%%%%%%%%%%%%%%%%%%%%%%%%%%%%%%%%%%%%%%%%%%%%%%%%%%%%%%%%%%%
