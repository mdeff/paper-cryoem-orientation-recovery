% ====================================
% PACKAGES
% ====================================

% ready for submission
%\usepackage{neurips_2021}
% to compile a preprint version, e.g., for submission to arXiv, add add the
% [preprint] option:
% \usepackage[preprint]{neurips_2021}
% to compile a camera-ready version, add the [final] option, e.g.:
%     \usepackage[final]{neurips_2021}
% to avoid loading the natbib package, add option nonatbib:
%    \usepackage[nonatbib]{neurips_2021}
\usepackage[nonatbib]{neurips_2021}

\usepackage[utf8]{inputenc} % allow utf-8 input
\usepackage[T1]{fontenc}    % use 8-bit T1 fonts
\usepackage{microtype}      % microtypography
\usepackage{amsfonts}
\usepackage{amsmath}
\usepackage{bm}
\usepackage[hidelinks]{hyperref}
\usepackage{graphicx}
\usepackage{mathtools}
\usepackage{enumitem}
\usepackage{setspace}
%\usepackage[tableposition=top]{caption}  % proper spacing for caption below tables
\usepackage{subcaption}
\usepackage[dvipsnames]{xcolor}
\usepackage{color}
\graphicspath{{figures/}}
\usepackage{upgreek}
\usepackage{cases}
\usepackage{arydshln}
\usepackage{wrapfig}
\usepackage{blkarray}
\usepackage{enumitem}
\usepackage{textcomp}  % silence gensym warnings
\usepackage{gensymb}  % for \degree
\usepackage{siunitx}  % for \num
%\sisetup{output-exponent-marker=\ensuremath{\mathrm{e}}}
\usepackage{booktabs}
\usepackage{longtable}
\usepackage[export]{adjustbox}% http://ctan.org/pkg/adjustbox
\usepackage[ruled,vlined]{algorithm2e}

%% ADDED BY LAURENE - DOUBLE SPACE
%\usepackage{setspace}
%%\onehalfspacing
%\doublespacing

% ====================================
% COMMANDS
% ====================================

\newcommand{\ie}{\textit{i.e.}}
\newcommand{\eg}{\textit{e.g.}}

\newcommand{\figref}[1]{Figure~\ref{fig:#1}}
\newcommand{\tabref}[1]{Table~\ref{tab:#1}}
%\newcommand{\secref}[1]{Section~\ref{sec:#1}}
\newcommand{\secref}[1]{\S\ref{sec:#1}}
\newcommand{\apxref}[1]{Appendix~\ref{apx:#1}}
%\newcommand{\eqnref}[1]{(\ref{eqn:#1})}
\newcommand{\eqnref}[1]{\eqref{eqn:#1}}
%\newcommand{\eqnref}[1]{equation~\eqref{eqn:#1}}

% comments
\newcommand{\todo}[1]{{\color[rgb]{.6,.1,.6}{#1}}}
\newcommand{\banjac}[1]{{\color[rgb]{.3,.5,.9}{#1}}}
\newcommand{\lau}[1]{{\textcolor{red}{#1}}}
\newcommand{\mdeff}[1]{{\color[rgb]{.8,.3,.2}{#1}}}

% ====================================
% MATH
% ====================================

%\newcommand\argmin[1]{\underset{#1}{\arg\;\min}}
\DeclareMathOperator*{\argmin}{arg\,min}

\newcommand\R{\mathbb{R}}
\newcommand\Rot{\mathbf{R}}
\newcommand\T{\mathbf{T}}
\newcommand\Or[0]{\mathrm{\mathbf{O}}}
\newcommand\SO[0]{\mathrm{\mathbf{SO}}}

\newcommand\x{\mathbf{x}}
\newcommand\p{\mathbf{p}}
\newcommand\f{\mathbf{f}}
\newcommand\G{\mathcal{G}}

\newcommand\bth[0]{{\boldsymbol{\theta}}}
\newcommand\bsx[0]{{\boldsymbol{x}}}

% ====================================
% TIKZ
% ====================================

\usepackage{tikz,pgfplots,pgfplotstable}
\usepgfplotslibrary{groupplots,fillbetween,colorbrewer,statistics}
\usetikzlibrary{intersections,calc,arrows,matrix,spy,pgfplots.statistics, pgfplots.colorbrewer}
\usepackage{color}
\definecolor{darkred}{rgb}{0.7,0,0}
\definecolor{darkgreen}{rgb}{0,0.5,0}
\definecolor{darkblue}{rgb}{0,0,0.7}
\definecolor{SkyBlue}{rgb}{0.53, 0.81, 0.92}
\pgfplotsset{compat=1.5.1, cycle list/Set1-3}
\usepackage{tikz-3dplot}
